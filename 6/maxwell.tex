\section{Maxwell's Equations}
Just a little identity on Laplacian of vector fields
\begin{proposition}
    We have
    $$\nabla^2\underline{F}=\nabla(\nabla\cdot\underline{F})-\nabla\times(\nabla\times\underline{F})$$
    where $(\nabla^2\underline{F})_i=\nabla^2\underline{F_i}$
\end{proposition}
\subsection{Introduction to Electromagnetism}
We have the electric field $\underline{E}=\underline{E}(\underline{x},t)$, the magnetic field $\underline{B}=\underline{B}(\underline{x},t)$, charge density $\rho=\rho(\underline{x},t)$ and current density $\underline{J}=\underline{J}(\underline{x},t)$.
The Maxwell's Equations state that
$$\begin{cases}
    \nabla\cdot\underline{E}=\epsilon_0^{-1}\rho\\
    \nabla\cdot\underline{B}=0\\
    \nabla\times\underline{E}+\partial\underline{B}/\partial t=0\\
    \nabla\times\underline{B}-\mu_0\epsilon_0\partial\underline{E}/\partial t=\mu_0\underline{J}
\end{cases}$$
where $\epsilon_0$ is the permitivity and $\mu_0$ is the permeability of free space with $\mu_0\epsilon_0=c^{-2}$ where $c$ is the speed of light.
Take the divergence of the fourth equation gives
$$0=\mu_0\epsilon_0\frac{\partial}{\partial t}(\nabla\cdot\underline{E})+\mu_0\nabla\cdot\underline{J}\implies 0=\frac{\partial\rho}{\partial t}+\nabla\cdot\underline{J}$$
\subsection{Integral Forms of Maxwell's Equations}
If we integrate then first equation and use divergence theorem to integrate the electric field over the flux of the boundary of a volume, we get
$$\int_{\partial V}\underline{E}\cdot\mathrm d\underline{S}=\int_V\nabla\cdot\underline{E}\,\mathrm dV=\frac{1}{\epsilon_0}\int_V\rho\,\mathrm dV=\frac{Q}{\epsilon_0}$$
where $Q$ is the total charge of the volume $V$.
Do exactly the same thing with th second equation then gives
$$\int_{\partial V}\underline{B}\cdot\mathrm d\underline{S}=\int_V\nabla\cdot\underline{B}\,\mathrm dV=0$$
which is saying that there is no magnetic monopoles since we cannot have a singular pole that emit magnetic field out of a volume.
In fact, if somewhere there exists a magnetic monopole, then charges are necessarily quantized.\\
As for the third equation, we have
$$\oint_{\partial S}\underline{E}\cdot\mathrm d\underline{x}=\int_S\nabla\times\underline{E}\cdot\mathrm d\underline{S}=-\frac{\mathrm d}{\mathrm dt}\int_S\underline{B}\,\mathrm d\underline{S}$$
So change in magnetic flux induces electric field.
Similarly, in the fourth equation,
$$\oint_{\partial S}\underline{B}\cdot\mathrm d\underline{x}=\int_S\nabla\times\underline{B}\cdot\mathrm d\underline{S}=\mu_0\epsilon_0\frac{\mathrm d}{\mathrm dt}\int_S\underline{E}\cdot\mathrm d\underline{S}+\mu_0\int_S\underline{J}\cdot\mathrm d\underline{S}$$
\subsection{Electromagnetic Waves}
In a free space, where $\rho=\underline{J}=\underline{0}$, then
\begin{align*}
    \nabla^2\underline{E}&=\nabla(\nabla\cdot\underline{E})-\nabla\times(\nabla\times\underline{E})\\
    &=0-\nabla\times\left(-\frac{\partial\underline{B}}{\partial t}\right)\\
    &=\frac{\partial}{\partial t}(\nabla\times\underline{B})\\
    &=\mu_0\epsilon_0\frac{\partial^2\underline{E}}{\partial t^2}\\
    \implies 0&=\nabla^2\underline{E}-\frac{1}{c^2}\frac{\partial^2\underline{E}}{\partial t^2}
\end{align*}
which is the wave equation of a wave with speed $c$.
If we do the same thing to $\underline{B}$, we obtain
\begin{align*}
    \nabla^2\underline{B}&=\nabla(\nabla\cdot\underline{B})-\nabla\times(\nabla\times\underline{B})\\
    &=0-\nabla\times\left(\frac{1}{c^2}\frac{\partial\underline{E}}{\partial t}\right)\\
    &=-\frac{1}{c^2}\frac{\partial}{\partial t}(\nabla\times\underline{E})\\
    &=\frac{1}{c^2}\frac{\partial^2\underline{B}}{\partial t^2}\\
    \implies 0&=\nabla^2\underline{B}-\frac{1}{c^2}\frac{\partial^2\underline{B}}{\partial t^2}
\end{align*}
\subsection{Electrostatics and Magnitostatics}
Assume that everything is time-independent, then $t$-derivatives are all $0$, which produces
$$\begin{cases}
    \nabla\cdot\underline{E}=\epsilon_0^{-1}\rho\\
    \nabla\cdot\underline{B}=0\\
    \nabla\times\underline{E}=0\\
    \nabla\times\underline{B}=\mu_0\underline{J}
\end{cases}$$
So if we work in $\mathbb R^3$ which is $2$-connected, we can write $\underline{E}=-\nabla\phi$ and $\underline{B}=\nabla\times\underline{A}$ for some $\phi,\underline{A}$.
$\phi$ is called the electric potential and $\underline{A}$ the magnetic potential.
So Maxwell's equations reduce to
$$\begin{cases}
    -\nabla^2\phi=\rho/\epsilon_0\\
    \nabla\times(\nabla\times\underline{A})=\mu_0\underline{J}
\end{cases}$$
The first one is called the Poisson's Equation.
\subsection{Gauge Invariance}
In a $2$-connected domain, we can always write $\underline{B}=\nabla\times\underline{A}$.
But note that the equation still holds by adding the gradient of some scalar function to $\underline{A}$, so $\underline{B}$ is invariant under $\underline{A}\mapsto\underline{A}+\nabla\chi$ for $\chi=\chi(\underline{x},t)$.
If we put the vector potential into the third equation,
$$\nabla\times\left( \underline{E}+\frac{\partial\underline{B}}{\partial t} \right)=0$$
So we can write $\underline{E}=-\nabla\phi-\partial\underline{B}/\partial t$, so we have
$$-\nabla^2\phi-\frac{\partial}{\partial t}(\nabla\cdot\underline{A})=\frac{\rho}{\epsilon_0}$$
And
$$\nabla\times(\nabla\times A)+\frac{1}{c^2}\nabla\left( \frac{\partial\phi}{\partial t} \right)+\frac{1}{c^2}\frac{\partial^2\underline{A}}{\partial t^2}=\mu_0\underline{J}$$
But by a known identity on curl of curl,
$$-\nabla^2\underline{A}+\frac{1}{c^2}\frac{\partial^2\underline{A}}{\partial t^2}+\nabla\left( \frac{1}{c^2}\frac{\partial\phi}{\partial t}+\nabla\cdot\underline{A} \right)=\mu_0\underline{J}$$
We now choose the scalar field $\chi$ such that
$$\frac{1}{c^2}\frac{\partial\phi}{\partial t}+\nabla\cdot\underline{A}=0$$
under $\underline{A}\mapsto\underline{A}+\nabla\chi$.
So we can get an equation similar to a wave equation
$$-\nabla^2\underline{A}+\frac{1}{c^2}\frac{\partial^2\underline{A}}{\partial t^2}=\mu_0\underline{J}$$
and
$$-\nabla^2\phi+\frac{1}{c^2}\frac{\partial^2\phi}{\partial t^2}=\frac{\rho}{\epsilon_0}$$
This trick is called the Lorenz gauge.