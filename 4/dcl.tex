\section{Divergence, Curl and Laplacian}
\subsection{Definitions}
\begin{definition}
    For a vector field $\underline{F}:\mathbb R^3\to\mathbb R^3$, we define the divergence
    $$\operatorname{div}\underline{F}=\nabla\cdot\underline{F}=\frac{\partial F_i}{\partial x_i}$$
    where the summation convention applies.
\end{definition}
\begin{definition}
    For a vector field $\underline{F}:\mathbb R^3\to\mathbb R^3$, we define the curl
    $$\operatorname{curl}\underline{F}=\nabla\times\underline{F}=\epsilon_{ijk}\frac{\partial F_k}{\partial x_j}\underline{e_i}$$
    where again the summation convention applies.
\end{definition}
\begin{definition}
    For a function $f:\mathbb R^3\to\mathbb R$, we define the Laplacian
    $$\Delta f=\nabla^2f=\nabla\cdot(\nabla f)=\frac{\partial^2 f}{\partial x_i\partial x_i}$$
    where yet again we have the summation convention.
\end{definition}
\begin{note}
    All these are in Cartesians.
\end{note}
\begin{example}
    Consider the vector field $\underline{F}(\underline{x})=\underline{x}$, then $\nabla\cdot\underline{F}=3$.
    Also $(\nabla\times\underline{F})_i=\epsilon_{ijk}\frac{\partial F_k}{\partial x_j}=\epsilon_{ijk}\delta_{jk}=0$, hence $\nabla\times\underline{F}=\underline{0}$.
\end{example}
\begin{proposition}
    We have the following identities:
    $$\nabla(fg)=(\nabla f)g+f(\nabla g)$$
    $$\nabla\cdot(f\underline{F})=(\nabla f)\cdot\underline{F}+f(\nabla\cdot\underline{F})$$
    $$\nabla\times (fF)=(\nabla f)\times\underline{F}+f(\nabla\times\underline{F})$$
    $$\nabla(\underline{F}\cdot\underline{G})=\underline{F}\times(\nabla\times\underline{G})+\underline{G}\times(\nabla\times\underline{F})+(\underline{F}\cdot\nabla)\underline{G}+(\underline{G}\cdot\nabla)\underline{F}$$
    $$\nabla\times(\underline{F}\times\underline{G})=\underline{F}(\nabla\cdot\underline{G})-\underline{G}(\nabla\cdot\underline{F})+(\underline{G}\cdot\nabla)\underline{F}-(\underline{F}\cdot\nabla)\underline{G}$$
    $$\nabla\cdot(\underline{F}\times\underline{G})=(\nabla\times\underline{F})\cdot\underline{G}-\underline{F}\cdot(\nabla\times\underline{G})$$
\end{proposition}
\begin{proof}
    Trivial.
\end{proof}
Of course we want, can can compute these three quantities in curvilinear coordinates, but we cannot do it directly since the basis vectors are not constant.
However we can expand everything and get
\begin{proposition}
    For a vector field $\underline{F}$ under a curvilinear coordinate $\underline{F}=F_u\underline{e_u}+F_v\underline{e_v}+F_w\underline{e_w}$,
    $$\nabla\cdot\underline{F}=\frac{1}{h_uh_vh_w}\sum_{u,v,w}^{\rm cyc}\frac{\partial}{\partial u}(h_vh_wF_u)$$
    $$\nabla\times\underline{F}=\sum_{u,v,w}^{\rm cyc}\frac{1}{h_vh_w}\left( \frac{\partial}{\partial v}(h_wF_w)-\frac{\partial}{\partial w}(h_vF_v) \right)\underline{e_u}$$
    And for a scalar function $f$,
    $$\nabla^2f=\frac{1}{h_uh_vh_w}\sum_{u,v,w}^{\rm cyc}\frac{\partial}{\partial u}\left(\frac{h_uh_w}{h_u}\frac{\partial f}{\partial u}\right)$$
\end{proposition}
\begin{proof}
    Trivial calculations.
\end{proof}
If one is bored, one can try and find the formulas for cylindral and spherical coordinates:
$$\nabla^2f=\frac{1}{\rho}\frac{\partial}{\partial\rho}\left( \rho\frac{\partial f}{\partial\rho} \right)+\frac{1}{\rho^2}\frac{\partial^2f}{\partial \phi^2}+\frac{\partial^2f}{\partial z^2}$$
$$\nabla^2f=\frac{1}{r^2}\frac{\partial}{\partial r}\left( r^2\frac{\partial f}{\partial r} \right)+\frac{1}{r^2\sin\theta}\frac{\partial}{\partial\theta}\left( \sin\theta\frac{\partial f}{\partial\theta} \right)+\frac{1}{r^2\sin\theta}\frac{\partial^2f}{\partial\phi^2}$$
The reason why we need these notions is for the generalization of fundamental Theorem of Calculus to general integrals, where some of these operators will be used as a substituent of derivative.\\
The reason we need Laplacians is that the PDE $\nabla^2f=0$, whose solutions are called harmonic functions, is pretty important.
One of their properties that once they are twice differentiable (so as to let the equation make sense), then they are infinitely differentiable.
Even better, they are all analytic, i.e. can be expressed in terms of power series.
\subsection{Relationships between the Operators}
\begin{proposition}
    Let $f:\mathbb R^3\to\mathbb R$ and $\underline{F}:\mathbb R^3\to\mathbb R^3$, then $\nabla\times\nabla f=0$ and $\nabla \cdot(\nabla\times\underline{F})=0$.
\end{proposition}
\begin{proof}
    Trivial.
\end{proof}
Hence if $\underline{F}$ is conservative, then it has zero curl.
The reverse implication is true when the domain is simply connected.
For example, if we take $\mathbb R^3\setminus\{(0,0,z):z\in\mathbb R\}$ as our domain, then this is not simply connected, but $\mathbb R^3\setminus\{(0,0,0)\}$ is.\\
If there exists vector fields $\underline{A}$ such that $\underline{F}=\nabla\times\underline{A}$, we say $\underline{A}$ is a vector potential of $\underline{F}$.
So if $\nabla\cdot\underline{F}=0$, we say $\underline{F}$ is solenoidal.
The existence of a vector potential for $\underline{F}$ implies $\underline{F}$ is solenoidal.
The reverse implication is true when the domain is $2$-connected, that is, it is simply connected and the second homotopy group is trivial.
For example, $\mathbb R^3$ is $2$-connected but $\mathbb R^3\setminus\{(0,0,0)\}$ is not.