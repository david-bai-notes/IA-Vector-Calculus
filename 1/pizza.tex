\subsection{Bonus: Gaussian Curvature and Pizza}
Choose a normal of a surface and consider planes containing that normal.
We can draw many curves on the surface now by considering the intersection of the planes and the surface and measure their curvatures at that particular point.
\begin{definition}
    The Gaussian curvature is defined as
    $$K_G=K_{\rm max} K_{\rm min}$$
    where $K_{\rm max}$ and $K_{\rm min}$ are the maximal and minimal curvatures of such curves.
\end{definition}
For example, a flat piece of paper has $K_G=0$.
Of course, we can define this much more rigorously, but that is out of the scope of this course.
For this definition of surface curvature, Gauss proved that:
\begin{theorem}[Theorema Egregium]
    The Gaussian curvature is invariant under isometries.
\end{theorem}
So it is like when you bend a pizza isometrically, since it still has Gaussian curvature $0$ as it had before, the pizza has to be flopped up so as to be eaten.