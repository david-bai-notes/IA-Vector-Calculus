\subsection{Surface Integrals}
\begin{definition}
    Consider a map $f:\mathbb R^3\to\mathbb R$, then we can define a surface by $\{\underline{x}:f(\underline{x})=0\}$.
\end{definition}
In this case, the normal to the surface at $\underline{x}$ is $\nabla f(\underline{x})$.
\begin{definition}
    A surface thus defined is called regular if $\nabla f\neq \underline{0}$ everywhere on the surface.
\end{definition}
\begin{example}
    Consider $f(x,y,z)=x^2+y^2+z^2-1$, then it defines the unit sphere $S^2$.
    Note that $\nabla f=(2x,2y,2z)^\top$ which is certainly normal to $S^2$.
    It is also regular.\\
    In spherical polars, it is in the form $f(r,\theta,\phi)=r^2-1$, so $\nabla f=2r\underline{e_r}=2\underline{x}$.
\end{example}
Some surfaces have a boundary, for example a hemisphere.
In this case we write $\partial S$ to be the boundary of $S$.
In particular the boundary of a hemisphere defined by $x^2+y^2+z^2-1=0$ and $z=0$ is the unit circle in the $x-y$ plane.
If a surface does not have a boundary, we say the boundary is empty.
In this case, we call such a surface closed.\\
Often easiest is to give a local coordinate $u,v$, so $S=\{\underline{x}=\underline{x}(u,v)\}$.
In this case, we define a normal as
$$\frac{\partial\underline{x}}{\partial u}\times\frac{\partial\underline{x}}{\partial v}\left/\middle\|\frac{\partial\underline{x}}{\partial u}\times\frac{\partial\underline{x}}{\partial v}\right\|$$
assuming it is well-defined.
For regular surfaces, this is always well-defined.
It can define the normal consistenly (in terms of sign) or smoothly if the surface is orientable (which we may not define rigourously, sadly).
If the surface is indeed orientable, we use the convention for the orientation of the boundary curve that when we moving along the boundaries, normal vectors are on our left.
\begin{example}
    Consider the hemisphere again using spherical polars $S=\{(\cos\phi\sin\theta,\sin\phi\sin\theta,\cos\theta):\theta\in[0,\pi/2],\phi\in[0,2\pi]\}$
    By calculation we get exactly the vector $\underline{e_r}$ as the normal.
\end{example}
To calculate the area of a surface, we want to partition the surface by a rectangularization of the $u-v$ plane.
So the area of the piece that might look like a parallelogram when we zoom in have an approximated area of $\delta u\delta v\|(\partial \underline{x}/\partial u)\times(\partial \underline{x}/\partial v)\|$, so the area is (from intuition):
\begin{definition}
    The area of the surface $S$ is
    $$\int_S\mathrm dS=\int_S|\mathrm d\underline{S}|=\int_S\left\|\frac{\partial\underline{x}}{\partial u}\times\frac{\partial\underline{x}}{\partial v}\right\|\,\mathrm du\,\mathrm dv$$
\end{definition}
Where we write $\mathrm d\underline{S}=\underline{n}\,\mathrm dS$.
\begin{example}
    We (yet again) look at the hemisphere paramterized by $S=\{(R\cos\phi\sin\theta,R\sin\phi\sin\theta,R\cos\theta):\theta\in[0,\pi/2],\phi\in[0,2\pi]\}$, so
    $$\int_S\mathrm dS=\int_S|\mathrm d\underline{S}|=\int_SR^2\sin\theta\,\mathrm d\theta\,\mathrm d\phi=2\pi R^2$$
\end{example}
We want to use similar method to define the flux integral, which is like the amount of fluid passing though the surface $S$ in unit time.
\begin{definition}
    We define the integral of $f:\mathbb R^3\to\mathbb R$ to be
    $$\int_Sf\,\mathrm dS=\iint_Sf(\underline{x}(u,v))\left\|\frac{\partial\underline{x}}{\partial u}\times\frac{\partial\underline{x}}{\partial v}\right\|\,\mathrm du\,\mathrm dv$$
\end{definition}
Suppose $S:\underline{x}=\underline{x}(u,v),S':\underline{\tilde{x}}=\underline{\tilde{x}}(\tilde{u},\tilde{v})$ are two different parameterizations of the same surface $S$, then we have $\underline{x}(u,v)=\underline{\tilde{x}}(\tilde{u}(u,v),\tilde{v}(u,v))$, where we assume that $\tilde{u},\tilde{u}$ are smooth bijections with smooth inverse.
So we have, by calculus,
$$\frac{\partial\underline{x}}{\partial u}\times\frac{\partial\underline{x}}{\partial v}=\frac{\partial\underline{\tilde{x}}}{\partial\tilde{u}}\times\frac{\partial\underline{\tilde{x}}}{\partial\tilde{v}}\frac{\partial (\tilde{u},\tilde{v})}{\partial (u,v)}$$
So
\begin{align*}
    \int_S f\,\mathrm dS&=\iint_Sf(\underline{x}(u,v))\left\|\frac{\partial\underline{x}}{\partial u}\times\frac{\partial\underline{x}}{\partial v}\right\|\,\mathrm du\,\mathrm dv\\
    &=\iint_{S}f(\underline{\tilde{x}}(\tilde{u}(u,v),\tilde{v}(u,v)))\left\|\frac{\partial\underline{\tilde{x}}}{\partial\tilde{u}}\times\frac{\partial\underline{\tilde{x}}}{\partial\tilde{v}}\frac{\partial (\tilde{u},\tilde{v})}{\partial (u,v)}\right\|\,\mathrm du\,\mathrm dv\\
    &=\iint_{S}f(\underline{\tilde{x}}(\tilde{u}(u,v),\tilde{v}(u,v)))\left\|\frac{\partial\underline{\tilde{x}}}{\partial\tilde{u}}\times\frac{\partial\underline{\tilde{x}}}{\partial\tilde{v}}\right\|\left|\frac{\partial (\tilde{u},\tilde{v})}{\partial (u,v)}\right|\,\mathrm du\,\mathrm dv\\
    &=\iint_{S'}f(\underline{\tilde{x}}(\tilde{u},\tilde{v}))\left\|\frac{\partial\underline{\tilde{x}}}{\partial \tilde{u}}\times\frac{\partial\underline{\tilde{x}}}{\partial\tilde{v}}\right\|\,\mathrm d\tilde{u}\,\mathrm d\tilde{v}\\
    &=\int_{S'} f\,\mathrm dS'
\end{align*}
Just as the Fundamental Theorem of Calculus told us that the integration over a derivative depends only on its endpoints, the integral over a surface of some sort of derivative will only depend on the boundary of the surface.
\footnote{In fact, this is true in manifolds of even higher dimensions, which is known as Stokes' Theorem.}
Then, for a vector field $\underline{F}$, we define the flux integral of it over the surface $S$ by
$$\int_S\underline{F}\cdot\mathrm d\underline{S}=\int_S\underline{F}\cdot\underline{n}\,\mathrm dS$$