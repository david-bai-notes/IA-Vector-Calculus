\documentclass[a4paper]{article}

\usepackage{hyperref}

\newcommand{\triposcourse}{Vector Calculus}
\newcommand{\triposterm}{Lent 2020}
\newcommand{\triposlecturer}{Dr. A. Ashton}
\newcommand{\tripospart}{IA}

\usepackage{amsmath}
\usepackage{amssymb}
\usepackage{amsthm}
\usepackage{mathrsfs}

\usepackage{tikz-cd}

\theoremstyle{plain}
\newtheorem{theorem}{Theorem}[section]
\newtheorem{lemma}[theorem]{Lemma}
\newtheorem{proposition}[theorem]{Proposition}
\newtheorem{corollary}[theorem]{Corollary}
\newtheorem{problem}[theorem]{Problem}
\newtheorem*{claim}{Claim}

\theoremstyle{definition}
\newtheorem{definition}{Definition}[section]
\newtheorem{conjecture}{Conjecture}[section]
\newtheorem{example}{Example}[section]

\theoremstyle{remark}
\newtheorem*{remark}{Remark}
\newtheorem*{note}{Note}

\title{\triposcourse{}
\thanks{Based on the lectures under the same name taught by \triposlecturer{} in \triposterm{}.}}
\author{Zhiyuan Bai}
\date{Compiled on \today}

%\setcounter{section}{-1}

\begin{document}
    \maketitle
    This document serves as a set of revision materials for the Cambridge Mathematical Tripos Part \tripospart{} course \textit{\triposcourse{}} in \triposterm{}.
    However, despite its primary focus, readers should note that it is NOT a verbatim recall of the lectures, since the author might have made further amendments in the content.
    Therefore, there should always be provisions for errors and typos while this material is being used.
    \tableofcontents
    \section{Differential Geometry of Space Curves}
\subsection{Parameterized Curve by Arc Length}
\begin{definition}
    A parameterized curve $C$ is the image of a continuous map $[a,b]\to \mathbb R^3$ sending $t\mapsto \underline{x}(t)$.
    We say $C$ is a differentiable parameterized curve if each component $x_i(t)$ is differentiable.
    We say $C$ is regular if $\underline{x}^\prime(t)\neq\underline{0}$ for any $t$.\\
    A regular and differentiable curve is called smooth.
\end{definition}
Since it is an applied course (sadly), we will assert that our curve is as differentiable as we like.\\
To find the length of this curve, we partition the interval $[a,b]$ by $a=t_0<t_1<t_2<\cdots <t_{n-1}<t_n=b$.
We define the length $l(C,P)$ with respect to this partition $P$ to be
$$l(C,P)=\sum_{i=0}^{n-1}|\underline{x}(t_{i+1})-\underline{x}(t_i)|$$
By some applied-maths-intuition nonsense, we get that if we make the differences $t_{i+1}-t_i$ small enough, we are going to approach the length of the curve $C$, independent of the way we approach the limit.
So an applied mathematician will then set
\begin{definition}
    The length $l(C)$ of the curve $C$ is
    $$\lim_{t_{i+1}-t_i\to 0}l(C,P)$$
    which, as that applied mathematician will discover joyfully, equals
    $$\int_{a}^b|\underline{x}^\prime(t)|\,\mathrm dt$$
    Sometimes we write it as
    $$\int_C\mathrm ds$$
\end{definition}
Again by intuition we are gonna write $\mathrm ds=\sqrt{\sum_i\dot{x}_i^2}\mathrm dt=\sqrt{\sum_i\mathrm dx_i^2}$.
\begin{definition}
    We define
    $$\int_Cf\,\mathrm ds=\int_a^bf(\underline{x}(t))|\underline{\dot{x}}(t)|\,\mathrm dt$$
    for smooth curve $C$.
    And for piecewise smooth curve $C=C_1\cup C_2\cup\cdots\cup C_n$, we set
    $$\int_Cf\,\mathrm ds=\sum_{i=1}^n\int_{C_i}f\,\mathrm ds$$
\end{definition}
\begin{example}
    1. Let $C$ be a circle of radius $r>0$, so we can parameterize it by $(r\cos t,r\sin t,0), t\in [0,2\pi]$, and we unsurprisingly find that its length is $2\pi r$.\\
    2. Take $C$ be the same circle as in 1, we have
    $$\int_C x^2y\,\mathrm ds=\int_a^b (r\cos t)^2(r\sin t) r\,\mathrm dt=0$$
\end{example}
\begin{proposition}
    The way we define curve integrals is independent of parameterization.
\end{proposition}
\begin{example}
    If we parameterize the circle as $(r\cos(2t),r\sin(2t),0),t\in [0,\pi]$, we still get the same thing.
\end{example}
\begin{proof}
    Let $\underline{s_1}(t),\underline{s_2}(\tau),t\in[a,b],\tau\in[\alpha,\beta]$ be two different parameterizations of $C$, then there exists a function $\tau\to t(\tau)$ such that $\underline{s_1}(t(\tau))=\underline{s_2}(\tau)$.
    Assume that $\mathrm dt/\mathrm d\tau$ is nonzero and $t(\tau)$ is a differentiable, invertible, and have differentiable inverse, then we have
    $$\underline{s_2}^\prime(\tau)=\frac{\mathrm d\underline{s_1}\circ t}{\mathrm d\tau}=\underline{s_1}^\prime(t)t^\prime(\tau)$$
    If $\mathrm dt/\mathrm d\tau>0$, we have
    $$\int_\alpha^\beta|\underline{s_2}^\prime(\tau)|\,\mathrm d\tau=\int_\alpha^\beta|\underline{s_1}^\prime(t(\tau))|t^\prime(\tau)\,\mathrm d\tau=\int_a^b|\underline{s_1}^\prime(t)|\,\mathrm dt$$
    Similar for other cases.
\end{proof}
We now know the arc length is
$$s(t)=\int_{t_0}^t|\underline{\dot{x}}(u)|\,\mathrm du$$
On a regular curve, $\mathrm ds/\mathrm dt=|\underline{\dot{r}}(t)|>0$, this tells us that we can indeed parameterize each regular curve wrt arc length.
This is done by observing $\mathrm dt/\mathrm ds=1/|\underline{\dot{r}}(t)|$ which means we can write $\underline{r}(s)=\underline{r}(t(s))$, where we have
$$\frac{\mathrm d\underline{r}}{\mathrm ds}=\frac{\underline{\dot{r}}(t)}{|\underline{\dot{r}}(t)|}$$
which is a unit vector.
Therefore,
\begin{lemma}
    Any smooth curve $C$ has a parameterization $\underline{r}(s)$ such that
    $$\left|\frac{\mathrm dr}{\mathrm ds}\right|\equiv 1$$
\end{lemma}
\begin{proof}
    Followed from above.
\end{proof}
\subsection{Curvature and Torsion}
Throughout this section, we are only interested in smooth curves $C$ parameterized by arc length $\underline{r}(s)$.
\begin{definition}
    The tangent vector is defined as $\underline{t}(s)=\underline{r}^\prime(s)$.
\end{definition}
Note that $\underline{t}$ is always unit as it is an arc length parameterization.
\begin{definition}
    The curvature of $\underline{r}(s)$ is defined as $\kappa(s)=|\underline{t}^\prime(s)|=|\underline{r}^{\prime\prime}(s)|$.
\end{definition}
Note that if we differentiate $\underline{t}\cdot\underline{t}=1$, then we have $\underline{t}\cdot\underline{t}^\prime=0$.
This shows that the unit vector in the direction $t^\prime$ has a geometric interpretation as the normal to a curve, so we define
\begin{definition}
    The principal normal $\underline{n}$ is the (unit) vector such that $\underline{t}^\prime=\kappa\underline{n}$.
\end{definition}
Naturally, when we already have a pair of orthonormal vectors in $\mathbb R^3$, adding a third one seems to be the next step to do.
\begin{definition}
    In $\mathbb R^3$, the binormal $\underline{b}$ is defined as $\underline{b}=\underline{t}\times\underline{n}$.
\end{definition}
Then the vectors $\underline{t},\underline{n},\underline{b}$ form an orthonormal basis for $\mathbb R^3$.
Again we have $\underline{b}\cdot\underline{b}^\prime=0$ as $\underline{b}$ is unit.
But we also have $\underline{t}\cdot\underline{b}=0$, we get $\underline{t}\cdot\underline{b}^\prime=0$.
Hence $\underline{n},\underline{b}^\prime$ are parallel.
\begin{definition}
    The torsion $\tau$ is defined as such that $\underline{b}^\prime=-\tau\underline{n}$.
\end{definition}
So we have got there
$$\begin{cases}
    \underline{t}^\prime=\kappa\underline{n}=\kappa(\underline{b}\times\underline{t})\\
    \underline{b}^\prime=-\tau\underline{n}=\tau(\underline{t}\times\underline{b})
\end{cases}$$
Intuitively and truthfully
\begin{proposition}
    The curvature and torsion uniquely defines a curve up to rigid motion.
\end{proposition}
\begin{proof}
    Picard-Lindel\"of Theorem.
\end{proof}
The Taylor expansion of $\underline{r}(t)$ around $0$ shows
$$\underline{r}(s)=\underline{r}(0)+s\underline{t}(0)+\frac{1}{2}s^2\kappa\underline{n}(0)+o(s^2)$$
Now we turn to consider the circle of best fit around $\underline{r_0}$.
Parameterize the circle (with radiu $r$) by and expand to see that the second order term is somewhat like $s^2\underline{n}/(2r)$, so it is natural to define
\begin{definition}
    The radius of curvature is defined as $r=1/\kappa$.
\end{definition}
    \subsection{Bonus: Gaussian Curvature and Pizza}
Choose a normal of a surface and consider planes containing that normal.
We can draw many curves on the surface now by considering the intersection of the planes and the surface and measure their curvatures at that particular point.
\begin{definition}
    The Gaussian curvature is defined as
    $$K_G=K_{\rm max} K_{\rm min}$$
    where $K_{\rm max}$ and $K_{\rm min}$ are the maximal and minimal curvatures of such curves.
\end{definition}
For example, a flat piece of paper has $K_G=0$.
Of course, we can define this much more rigorously, but that is out of the scope of this course.
For this definition of surface curvature, Gauss proved that:
\begin{theorem}[Theorema Egregium]
    The Gaussian curvature is invariant under isometries.
\end{theorem}
So it is like when you bend a pizza isometrically, since it still has Gaussian curvature $0$ as it had before, the pizza has to be flopped up so as to be eaten.
    \section{Coordinates, Differentials and Gradients}
\subsection{Differentials and First-Order Changes}
Recall that if $f=f(u_1,\ldots,u_n)$, then we write $\mathrm df=(\partial f/\partial u_i)\,\mathrm du_i$ (the summation convention is being used).
Those $\mathrm du_i$ are formal objects called differential forms which are quite abstract geometrical notions that are way beyond the scope of this course.
These differential forms are taken as linearly independent the same way as vectors are.
Similarly, if $\underline{x}=\underline{x}(u_1,u_2,\ldots,u_n)$, then $\mathrm d\underline{x}=(\partial \underline{x}/\partial u_i)\,\mathrm du_i$.
\begin{example}
    If $f(u,v,w)=u^2-v^2+e^w$, then $\mathrm df=2u\,\mathrm du-2v\,\mathrm dv+e^w\,\mathrm dw$.
    If $\underline{x}=(u^2,v^2,w^2)^\top$, then $\mathrm d\underline{x}=(2u\,\mathrm du,2v\,\mathrm dv,2w\,\mathrm dw)^\top$.
\end{example}
Differential forms give a great tool to describe first-order changes.
If we perturb a multivariable function $f(u_1,\ldots,u_n)$, then we can have
$$f(u_1+\epsilon_1,\ldots,u_n+\epsilon_n)=f(u_1,\ldots,u_n)+\frac{\partial f}{\partial u_i}\epsilon_i+o(\|\underline{\epsilon}\|)$$
We can get the chain rule ``for free'' by using this notion.
Suppose we change our coordinates by $v_i=v_i(u_1,\ldots,u_n)$ and $F(u_1,\ldots,u_n)=f(v_1,\ldots,v_n)$, so
$$\frac{\partial F}{\partial u_i}\,\mathrm du_i=dF=df=\frac{\partial f}{\partial v_j}\,\mathrm dv_j=\frac{\partial f}{\partial v_j}\frac{\partial v_j}{\partial u_i}\,\mathrm du_i$$
Therefore
\begin{theorem}
    $$\frac{\partial F}{\partial u_i}=\frac{\partial f}{\partial v_j}\frac{\partial v_j}{\partial u_i}$$
\end{theorem}
Note that the summation convention is implicitly used.
\subsection{Coordinates in Line Elements}
Say $u,v$ are coordinates for $\mathbb R^2$ by relating them to Cartesians in the form $x=x(u,v),y=y(u,v)$ such that these smooth functions can be inverted smoothly to give $u=u(x,y),v=v(x,y)$.
\begin{example}
    Consider the polar coordinate $(r,\theta)$ with the relationship $x=r\cos\theta,y=r\sin\theta$.
    We can invert to have $r=\sqrt{x^2+y^2},\tan\theta=x/y$.
    \footnote{This is not quite invertible at $(x,y)=(0,0)$.
    Just saying.}
\end{example}
\begin{example}
    1. The Cartesian coordinate in $\mathbb R^2$ is $\underline{x}=\underline{x}(x,y)=(x,y)^\top$.
    \footnote{For future reference, any column vector in this course are implicitly written as per the standard basis unless otherwise specified.}
    Note that $\underline{x}_x,\underline{x}_y$ give the standard basis, so $\mathrm d\underline{x}=(\mathrm dx,\mathrm dy)^\top$.\\
    2. The polar coordinate defined above, $\underline{x}=(r\cos\theta,r\sin\theta)^\top$ has
    $$\underline{x}_r=(\cos\theta,\sin\theta),\underline{x}_\theta=(-r\sin\theta,r\cos\theta)$$
    which becomes an orthonormal basis (which depends on $(r,\theta)$) if we normalize.
    So the line element has
    $$\mathrm d\underline{x}=\begin{pmatrix}
        \cos\theta\\\sin\theta
    \end{pmatrix}\,\mathrm dr+r\begin{pmatrix}
        -\sin\theta\\\cos\theta
    \end{pmatrix}\,\mathrm d\theta$$
    So small change in the line element can result in a large change in the coordinate $\mathrm d\theta$.
    The basis vectors above give the rotation basis.
\end{example}
\begin{definition}
    We say $u,v,w$ are set of orthogonal curvilinear coordinates for $\mathbb R^3$ if the unit vectors $\underline{e}_u=\underline{x}_u/\|\underline{x}_u\|,\underline{e}_v=\underline{x}_v/\|\underline{x}_v\|,\underline{e}_w=\underline{x}_w/\|\underline{x}_w\|$ always forms a right-handed system of orthonormal vectors.
\end{definition}
\begin{definition}
    The scale factors are
    $$h_{u}=\left|\frac{\partial \underline{x}}{\partial u}\right|,h_{v}=\left|\frac{\partial \underline{x}}{\partial v}\right|,h_{w}=\left|\frac{\partial \underline{x}}{\partial w}\right|$$
\end{definition}
So the factors are the scaling factors for a little change in the corresponding coordinates.
\begin{definition}
    The cylindral polar coordinates are $(\rho,\phi,z)$ where
    $$\underline{x}=\underline{x}(\rho,\phi,z)=(\rho\cos\phi,\rho\sin\phi,z)^\top$$
\end{definition}
\begin{definition}
    The spherical polar coordinates are $(r,\theta,\phi)$ where
    $$\underline{x}=\underline{x}(r,\theta,\phi)=(r\cos\phi\sin\theta, r\sin\phi\sin\theta,r\cos\theta)$$
\end{definition}
\subsection{The Gradient Operator}
For $f:\mathbb R^3\to\mathbb R$, define the gradientof $f$, $\nabla f$, by
$$f(\underline{x}+\underline{h})=f(\underline{x})+\nabla f(\underline{x})\cdot\underline{h}+o(\underline{h})$$
\begin{definition}
    The directional derivative of $f$ in direction $\underline{v}$ by
    $$D_{\underline{v}}f(\underline{x})=\lim_{t\to 0}\frac{f(\underline{x}+t\underline{v})-f(\underline{x})}{t}$$
    So $f(\underline{x}+t\underline{v})=f(\underline{x})+tD_{\underline{v}}f(\underline{x})+o(t)$.
\end{definition}
So if we let $\underline{h}=t\underline{v}$, we have
$$f(\underline{x}+t\underline{v})=f(\underline{x})+\nabla f(\underline{x})\cdot t\underline{v}+o(t)$$
So we have $\nabla f(\underline{x})\cdot \underline{v}=D_{\underline{v}}f(\underline{x})$.
By Cauchy-Schwartz, to maximize a dot product we will have to make the two vectors parallel, so
\begin{proposition}
    $\nabla f(\underline{x})$ is the direction of greatest increase of $f$ at $\underline{x}$.
\end{proposition}
\begin{proof}
    Cauchy-Schwartz.
\end{proof}
\begin{example}
    1. $f(\underline{x})=|\underline{x}|^2/2$, so
    $$f(\underline{x}+\underline{h})=(\underline{x}+\underline{h})\cdot(\underline{x}+\underline{h})/2=f(\underline{x})+\underline{x}\cdot\underline{h}+o(\underline{h})$$
    Hence $\nabla f(\underline{x})=\underline{x}$.\\
    2. For generic curve $t\mapsto \underline{x}(t)$ and a function $F$, we want to evaluate $(F\circ \underline{x})^\prime$, so
    $$F(\underline{x}(t+\delta t))=F(\underline{x}(t)+\underline{\delta x})=F(\underline{x}(t))+\nabla f(\underline{x}(t))\cdot\underline{\delta x}+o(\underline{\delta x})$$
    where $\underline{\delta x}=\underline{x}(t+\delta t)-\underline{x}(t)=t\underline{x}^\prime(t)+o(t)$.
    So plugging it in we have
    $$\frac{\mathrm dF}{\mathrm dt}=\nabla F(\underline{x}(t))\cdot \underline{x}^\prime(t)$$
    3. Consider a surface in $\mathbb R^3$ by $S=\{\underline{x}:f(\underline{x})=0\}$ where $f:\mathbb R^3\to\mathbb R$.
    Take curve $t\mapsto \underline{x}(t)$ such that $\underline{x}\in S$ for all $t$, then $0=\underline{x}^\prime(t)\cdot\nabla F(\underline{x}(t))$, so $\nabla F$ is perpendicular to the tangent to the curve.
    Hence necessarily $\nabla F$ is the normal to $S$.
\end{example}
\subsection{Computing the Gradient}
For general orthogonal curvilinear coordinate, it might be hard to calculate $\nabla f$ since we do not usually know how to change the coordinates to accomodate the change $\underline{h}$.\\
But it is easy in Cartesians.
Just evaluating the directional derivatives at the basis vectors reveals:
\begin{proposition}
    $$\nabla f=\begin{pmatrix}
        \partial f/\partial x\\
        \partial f/\partial y\\
        \partial f/\partial z
    \end{pmatrix}$$
\end{proposition}
\begin{proof}
    We have
    $$(\nabla f(\underline{x}))_i=\nabla f(\underline{x})\cdot\underline{e_i}=D_{\underline{e_i}}f(\underline{x})$$
    As desired.
\end{proof}
\begin{example}
    Again we take $f(\underline{x})=|\underline{x}|^2/2$, then $(\nabla f(\underline{x}))_i=x_i$, so $\nabla f(\underline{x})=\underline{x}$ as before.
\end{example}
In Cartesians, we know the line elements $\mathrm d\underline{x}=\mathrm dx_i\underline{e_i}$ which allows us to calculate easily.
But we have $\mathrm df=D_{\underline{e_i}}f\,\mathrm dx_i$ in any coordinate.
So immediately we have
\begin{proposition}
    $\mathrm df=\nabla f\cdot\mathrm d\underline{x}$
\end{proposition}
which is coordinate independent.
\begin{proof}
    Just calculate the right hand side.
\end{proof}
\begin{proposition}
    Let $u,v,w$ be a set of curvilinear coordinates, then we have
    $$\nabla f=\frac{1}{h_u}\frac{\partial f}{\partial u}\underline{e_u}+\frac{1}{h_v}\frac{\partial f}{\partial v}\underline{e_v}+\frac{1}{h_w}\frac{\partial f}{\partial w}\underline{e_w}$$
\end{proposition}
\begin{proof}
    $\mathrm df=\nabla f\cdot\mathrm d\underline{x}$ independent of coordinate.
    We also know that $\mathrm d\underline{x}=h_u\underline{e_u}\,\mathrm du+h_v\underline{e_v}\,\mathrm dv+h_w\underline{e_w}\,\mathrm dw$.
    Write $\nabla f=(\nabla f)_u\underline{e_u}+(\nabla f)_v\underline{e_v}+(\nabla f)_w\underline{e_w}$, then
    $$\nabla f\cdot \mathrm d\underline{x}=h_u(\nabla f)_u\,\mathrm du+h_v(\nabla f)_v\,\mathrm dv+h_w(\nabla f)_w\,\mathrm dw$$
    In addition,
    $$\mathrm df=\frac{\partial f}{\partial u}\,\mathrm du+\frac{\partial f}{\partial v}\,\mathrm dv+\frac{\partial f}{\partial w}\,\mathrm dw$$
    But they are equal.
    Since $\mathrm du,\mathrm dv,\mathrm dw$ are linearly independent, we have
    $$(\nabla f)_u=\frac{1}{h_u}\frac{\partial f}{\partial u},(\nabla f)_v=\frac{1}{h_v}\frac{\partial f}{\partial v},(\nabla f)_w=\frac{1}{h_w}\frac{\partial f}{\partial w}$$
    As desired.
\end{proof}
\begin{example}
    1. For cylindral coordinates $(\rho,\phi,z)$, we have
    $$\nabla f=\frac{\partial f}{\partial\rho}\underline{e_\rho}+\frac{1}{\rho}\frac{\partial f}{\partial\phi}\underline{e_\phi}+\frac{\partial f}{\partial z}\underline{e_z}$$
    Then in the previous example where $f(\underline{x})=|\underline{x}|^2/2=(\rho^2+z^2)/2\implies\nabla f=\rho\underline{e_\rho}+z\underline{e_z}=\underline{x}$\\
    2. For spherical coordinates $(r,\theta,\phi)$, we can do the same thing,
    $$\nabla f=\frac{\partial f}{\partial r}\underline{e_r}+\frac{1}{r}\frac{\partial f}{\partial\theta}\underline{e_\theta}+\frac{1}{r\sin\theta}\frac{\partial f}{\partial \phi}\underline{e_\phi}$$
    Using the same example $f(\underline{x})=r^2/2\implies \nabla f=r\underline{e_r}=\underline{x}$.
\end{example}
\begin{note}
    We talked about the functions about position vectors under different coordinates,
    $$f(\underline{x})=f(\underline{x}(x,y,z))=f(\underline{x}(r,\theta,\phi))$$
    So when we are talking about $f$, sometimes we are telling
    $$\tilde{f}(x,y,z)=f(\underline{x}(x,y,z)),\tilde{\tilde{f}}(r,\theta,\rho)=f(\underline{x}(r,\theta,\rho))$$
    Or other coordinate we might find interesting.
    We are actually talking about a pullback here which might be clear in a couple of years' time.
\end{note}
    \section{Integration along Lines, Surfaces, and Volumes}
\subsection{Line Integrals}
\begin{definition}
    For a vector field $\underline{F}(\underline{x})$ and a curve $C$ where $\underline{x}(t)$ travels through for $t\in [a,b]$, we define the line integral
    $$\int_C\underline{F}\cdot\mathrm d\underline{x}=\int_a^b\underline{F}(\underline{x}(t))\cdot\frac{\mathrm d\underline{x}}{\mathrm dt}\,\mathrm dt$$
\end{definition}
\begin{example}
    Let
    $$\underline{F}=\begin{pmatrix}
        x^2y\\
        yz\\
        2zx
    \end{pmatrix},C_1:[0,1]\ni t\mapsto\begin{pmatrix}
        t\\
        t\\
        t
    \end{pmatrix},C_2:[0,1]\ni t\mapsto\begin{pmatrix}
        t\\
        t\\
        t^2
    \end{pmatrix}$$
    ...
    We have
    $$\int_{C_1}\underline{F}\cdot\mathrm d\underline{x}=\frac{5}{4},\int_{C_2}\underline{F}\cdot\mathrm d\underline{x}=\frac{13}{10}$$
    Hence in general line integrals between two points depends on path.
\end{example}
\begin{example}
    In cylindral polars $(\rho,\phi,z)$, we consider $\underline{F}=\rho z\underline{e_{\phi}}$.
    Consider the line $C:[0,2\pi]\ni t\mapsto (a\cos t,a\sin t,t)^\top$.
    So $\rho=a,\phi=z=t$
    We have $\underline{F}\cdot\mathrm d\underline{x}=\rho^2 z\mathrm d\phi$, therefore
    $$\int_C\underline{F}\cdot\mathrm d\underline{x}=\int_0^{2\pi}a^2t\,\mathrm dt=2a^2\pi^2$$
\end{example}
In those cases where $C$ is closed, we write
$$\oint_C\underline{F}\cdot\mathrm d\underline{x}=\int_C\underline{F}\cdot\mathrm d\underline{x}$$
This is sometimes called the circulation of $\underline{F}$ over $C$.
\subsection{Conservative Forces and Exact Differentials}
$\underline{F}\cdot\mathrm d\underline{x}$ is an example of a differential form.
\begin{definition}
    We say $\underline{F}\cdot\mathrm d\underline{x}$ is exact if $\underline{F}\cdot\mathrm d\underline{x}=\mathrm df$ for some scalar function $f$.
    Equivalently, the differential form is exact iff $\underline{F}=\nabla f$ for a scalar function $f$.
    In this case, we say $\underline{F}$ is conservative.
\end{definition}
\begin{proposition}
    If $\underline{F}\cdot\mathrm d\underline{x}$ is exact, then
    $$\oint_C\underline{F}\cdot\mathrm d\underline{x}=0$$
    for any closed $C$.
\end{proposition}
\begin{proof}
    By exactness, $\underline{F}=\nabla f$ for a scalar function $f$.
    Suppose $C: [a,b]\ni t\mapsto\underline{x}(t)$.
    $$\oint_C\underline{F}\cdot\mathrm d\underline{x}=\int_a^b\nabla f(\underline{x}(t))\cdot\underline{x}^\prime(t)\,\mathrm dt=\int_a^b\frac{\mathrm d}{\mathrm dt}f(\underline{x}(t))\,\mathrm dt=f(\underline{x}(b))-f(\underline{x}(a))=0$$
    since $\underline{x}(b)=\underline{x}(a)$.
\end{proof}
\begin{remark}
    Consider the cylindral coordinate $(\rho,\phi,z)$, suppose $\underline{F}\cdot\mathrm d\underline{x}=\mathrm d\phi$ and $C:[0,2\pi]\ni t\mapsto (\cos t,\sin t,0)^\top$, then by calculation we have
    $$\oint_C \underline{F}\cdot\mathrm d\underline{x}=2\pi\neq 0$$
    It does not work!
    The reason for this is that $\phi\in\mathbb R/2\pi\mathbb Z$.
    So we are doing here is taking $\phi$ as a multivalued function instead of an actual function.
\end{remark}
Suppose we have a set of curvilinear coordinate $(u,v,w)=(u_1,u_2,u_3)$, and we have $\underline{F}\cdot\mathrm d\underline{x}=\theta_i\,\mathrm du_i$ where $\theta_i=\underline{F}\cdot\mathrm d\underline{x}/\mathrm du_i$.
If $\underline{F}\cdot\mathrm d\underline{x}$ is exact and is the differential $\mathrm df$, then $\theta_i=\partial f/\partial u_i$, and
$$\frac{\partial \theta_i}{\partial u_j}=\frac{\partial^2 f}{\partial u_i\partial u_j}=\frac{\partial^2 f}{\partial u_j\partial u_i}=\frac{\partial \theta_j}{\partial u_i}$$
\begin{definition}
    If the above condition is met, we say the differential form $\underline{F}\cdot\mathrm d\underline{x}$ is closed.
\end{definition}
So exact differentials are all closed.
\begin{theorem}
    If the domain of the vector field is simply connected, then any closed differential is exact.
\end{theorem}
\begin{proof}
    Not gonna do it.
\end{proof}
\begin{example}
    1. The differential $y\,\mathrm dx-x\,\mathrm dy$ is not closed hence not exact.\\
    2. Consider $C:[t_1,t_2]\ni t\mapsto (f,g,h)$ where $f,g,h$ are completely unintelligible functions such that $(f,g,h)(t_1)=(f,g,h)(t_2)$, then whatever they are, we always have
    $$\oint_C 3x^2y\,\mathrm dx+x^3\,\mathrm dy=0$$
    As the integrand is exact.
\end{example}
\begin{theorem}
    Suppose $\underline{F}\cdot\mathrm\underline{x}=\mathrm df$, then consider $C$ from $\underline{a}$ from $\underline{b}$, then we have
    $$\int_C\underline{F}\cdot\mathrm\underline{x}=f(\underline{b})-f(\underline{a})$$
\end{theorem}
\begin{proof}
    Obvious.
\end{proof}
\begin{example}
    Suppose $\underline{F}=m\underline{\ddot{x}}$ and $C:[a,b]\ni t\mapsto\underline{x}(t)$, we have
    $$\int_C\underline{F}\cdot\mathrm d\underline{x}=m\int_a^b\underline{\ddot{x}}\cdot\underline{\dot{x}}\,\mathrm dt=\left.\frac{1}{2}m|\underline{\dot{x}}|^2\right|_a^b$$
    If $\underline{F}=-\nabla V$, then we have the conservation of energy:
    $$\left.\frac{1}{2}m|\underline{\dot{x}}|^2\right|_a^b=\int_C\underline{F}\cdot\mathrm d\underline{x}=-\left.V(\underline{x}(t))\right|^b_a$$
    So $V(\underline{x})+m|\underline{\dot{x}}|^2/2$ conserves.
\end{example}
\subsection{Integration over Areas}
We want to extend our definition of Riemannian integrals to $\mathbb R^2$.
To do it, we partition our region $D$ into small cells $A_{ij}$ with area $\delta A_{ij}$ diameter at most $\epsilon$ and pick points $(x_i,y_j)\in A_{ij}$.
\begin{definition}
    We thus define the area integral by
    $$\int_Df\,\mathrm dA=\lim_{\epsilon\to 0}\sum_{i,j}f(x_i,y_j)\delta A_{ij}$$
    We say this integral exists if this limit is independent of the choice of the partition $A_{ij}$.
\end{definition}
When the integral exists, the obivous choice is to split $D$ into rectangular cells and set $x_{i+1}=x_i+\delta x,y_{j+1}=y_j+\delta y$ such that $0<\delta x,\delta y<<\epsilon$.
Then we may fix $y$ first and take $\delta x\to 0$, and then do $\delta y\to 0$.
That is, we split the horizontal region by $\epsilon$-thin stripes and sum over the Riemann integrals in each stripe.
If we do this, then
$$\int_Df\,\mathrm dA=\int_Y\int_{x_y}f(x,y)\,\mathrm dx\,\mathrm dy$$
where $x_y=\{x:(x,y)\in D\}$.
If we do vertical stripes first, then we get stuff like
$$\int_Df\,\mathrm dA=\int_X\int_{y_x}f(x,y)\,\mathrm dy\,\mathrm dx$$
where $y_x=\{y:(x,y)\in D\}$.
In short, we seems to have $\mathrm dA=\mathrm dx\,\mathrm dy=\mathrm dy\,\mathrm dx$
\begin{theorem}[Fubini's Theorem]
    If the integral exists (or under suitable conditions), then
    $$\int_Df\,\mathrm dA=\int_Y\int_{x_y}f(x,y)\,\mathrm dx\,\mathrm dy=\int_X\int_{y_x}f(x,y)\,\mathrm dy\,\mathrm dx$$
\end{theorem}
\begin{example}
    $f(x)=xy^2$ and $D$ is the triangle joining $(0,0),(0,1),(1,0)$.
    Then we have
    $$\int_Df\,\mathrm dA=\int_0^1\int_{0}^{1-y}xy^2\,\mathrm dx\,\mathrm dy=\int_0^1\frac{(1-y)^2y^2}{2}\,\mathrm dy=\frac{1}{60}$$
    If we do $y$ (vertical slices) first,
    $$\int_Df\,\mathrm dA=\int_0^1\int_0^{1-x}xy^2\,\mathrm dy\,\mathrm dx=\int_0^1\frac{x(1-x)^3}{3}\,\mathrm dx=\frac{1}{60}$$
\end{example}
Recall that in the one dimensional case, we can do integrations by some magical substitutions.
Obviously we will wish to extend this technique to integrations over higher dimensions.
\begin{proposition}[Change of Variable]
    Let $x=x(u,v),y=y(u,v)$ be a smooth bijection $D\to D'$ with smooth inverse, then
    $$\iint_D f(x,y)\,\mathrm dx\,\mathrm dy=\iint_{D'}f(x(u,v),y(u,v))|J|\,\mathrm du\,\mathrm dv$$
    where
    $$J=\frac{\partial (x,y)}{\partial (u,v)}=\begin{vmatrix}
        x_u&x_v\\
        y_u&y_v
    \end{vmatrix}$$
    is the Jacobian.
\end{proposition}
\begin{proof}
    Partition $D$ using image of rectangular partition of $D'$.
    Then we have
    $$\int_Af\,\mathrm dA=\lim_{\epsilon\to 0}\sum_{i,j}f(x(u_i,v_j),y(u_i.y_j))\delta A_{ij}^{x,y}$$
    But $\delta A_{ij}^{x,y}\approx |J|\delta A_{ij}^{u,v}$ by considering the area as local parallelograms and expanding the Taylor series to first order.
    So we have this formula.
\end{proof}
\begin{example}
    Consider $x=\rho\cos\phi,y=\rho\sin\phi$ for $\rho\ge 0$, then $|J|=\rho$, hence $\mathrm dx\,\mathrm dy=\rho\,\mathrm d\rho\,\mathrm d\phi$.
    Take $D$ to be the region $x>0,y>0$, then this region is mapped to the region $\phi\in (0,\pi/2)$.
    So let
    $$I=\int_0^\infty e^{-x^2}\,\mathrm dx$$
    then
    $$I^2=\int_0^\infty\int_0^\infty e^{-x^2-y^2}\rho\,\mathrm dx\,\mathrm dy=\int_0^{\pi/2}\int_0^\infty e^{-\rho^2}\rho\,\mathrm d\rho\,\mathrm d\phi=\frac{\pi}{4}\implies I=\frac{\sqrt\pi}{2}$$
\end{example}
\subsection{Integration over Volumes}
For a bounded volume $V$ in $\mathbb R^3$, define sets $V_{ijk}$ having volume $\delta V_{ijk}$ which partition $V$ and each is contained in a ball of radius at most $\epsilon$.
Then we pick some $(x_i,y_j,z_k)$ in each cell $V_{ijk}$ and define the integral over the region $V$ as
$$\lim_{\epsilon\to 0^+}\sum_{i,j,k}f(x_i,y_j.z_k)\delta V_{ijk}$$
If we use a rectangular parallelopiped partition, we find that $\mathrm dV=\mathrm dx\,\mathrm dy\,\mathrm dz$ in any order (by Fubini).
\begin{example}
    1. Consider the domain to be the tetrahedron $V=\{(x,y,z)\in\mathbb R_{\ge 0}^3:x+y+z\le 1\}$.
    So
    $$\int_V\mathrm dV=\int_0^1\int_0^{1-x}\int_0^{1-x-y}\mathrm dz\,\mathrm dy\,\mathrm dx=\frac{1}{6}$$
    2. For a volumn $V$, we define the center of mass $\Delta_{\rm COM}$ by
    $$\Delta_{\rm COM}=\frac{1}{M}\int_V\rho\underline{x}\,\mathrm dV$$
    where $\rho$ is the density, $M=\rho V$ is the mass and the integral is by component.
    Consider the same tetrahedron as above and suppose $\rho=1$.
    Hence $M=1/6$, so
    $$\Delta_{\rm COM}=6\int_V1\begin{pmatrix}
        x\\
        y\\
        z
    \end{pmatrix}\,\mathrm dV=\frac{1}{4}\begin{pmatrix}
        1\\
        1\\
        1
    \end{pmatrix}$$
\end{example}
\begin{proposition}
    Let $\underline{x}=\underline{x}(u,v,w)$ (where $\underline{x}=(x,y,z)$), denote the smooth bijection with smooth inverse which connects the region $V$ in the $xyz$ space and $V'$ in the $uvw$ space, then
    $$\iiint_Vf(x,y,z)\,\mathrm dx\,\mathrm dy\,\mathrm dz=\iiint_{V'}f(x(u,v,w),y(u,v,w),z(u,v,w))|J|\,\mathrm du\,\mathrm dv\,\mathrm dw$$
    where
    $$J=\begin{vmatrix}
        x_u&x_v&x_w\\
        y_u&y_v&y_w\\
        z_u&z_v&z_w
    \end{vmatrix}$$
\end{proposition}
\begin{proof}
    Same (imprecise) idea.
\end{proof}
\begin{example}
    If we use cylindral polars, we find $\mathrm dx\,\mathrm dy\,\mathrm dz=\rho\,\mathrm d\rho\,\mathrm d\phi\,\mathrm dz$, and if we use spherical polars, we have $\mathrm dx\,\mathrm dy\,\mathrm dz=r^2\sin\theta\,\mathrm dr\,\mathrm d\theta\,\mathrm d\phi$.
\end{example}
\begin{example}
    1. Consider a sphere with radius $R$, we want to find its volume.
    If we do it with Cartesians, then
    $$\int_{-R}^R\int_{-\sqrt{R^2-z^2}}^{\sqrt{R^2-z^2}}\int_{-\sqrt{R^2-z^2-y^2}}^{\sqrt{R^2-z^2-y^2}}\mathrm dx\,\mathrm dy\,\mathrm dz=\frac{4}{3}\pi R^3$$
    after a lot of useless effort.\\
    So obviously we choose to use spherical polars, hence the volume is
    $$\int_0^{2\pi}\int_0^\pi\int_0^{R}r^2\sin\theta\,\mathrm dr\,\mathrm d\theta\,\mathrm d\phi=\frac{4}{3}\pi R^2$$
    after minimal effort.\\
    2. A ball of radius $b>0$ with cylinder with radius $a>0$ (and infinite length) with $a>0$ removed.
    So maybe we will use some cylindral polars, so the volume is
    $$\int_0^{2\pi}\int_a^b\int_{-\sqrt{b^2-\rho^2}}^{\sqrt{b^2-\rho^2}}\rho\,\mathrm dz\,\mathrm d\rho\,\mathrm d\phi=\frac{4}{3}\pi(b^2-a^2)^{3/2}$$
    which was easy.
\end{example}
    \subsection{Bonus: Lebesgue Integral}
If we want to find the area under a curve, we choose to slice up the horizontal region (domain) of the function (Riemannian integration).
But can we choose to slice up the vertical region that is the range of the function?
This is known as the Lebesgue integration.
Sometimes this is better.
For example if we want to integrate $f(x)=1_{\mathbb Q}$, Riemannian integration would not work if we want to evaluate
$$\int_0^1f(x)\,\mathrm dx$$
But if we do it by Lebesgue intgeral, but since $\mathbb Q$ is countable
\footnote{Hence of Lebesgue measure $0$.}
this integral is obviously $0$.
    \subsection{Surface Integrals}
\begin{definition}
    Consider a map $f:\mathbb R^3\to\mathbb R$, then we can define a surface by $\{\underline{x}:f(\underline{x})=0\}$.
\end{definition}
In this case, the normal to the surface at $\underline{x}$ is $\nabla f(\underline{x})$.
\begin{definition}
    A surface thus defined is called regular if $\nabla f\neq \underline{0}$ everywhere on the surface.
\end{definition}
\begin{example}
    Consider $f(x,y,z)=x^2+y^2+z^2-1$, then it defines the unit sphere $S^2$.
    Note that $\nabla f=(2x,2y,2z)^\top$ which is certainly normal to $S^2$.
    It is also regular.\\
    In spherical polars, it is in the form $f(r,\theta,\phi)=r^2-1$, so $\nabla f=2r\underline{e_r}=2\underline{x}$.
\end{example}
Some surfaces have a boundary, for example a hemisphere.
In this case we write $\partial S$ to be the boundary of $S$.
In particular the boundary of a hemisphere defined by $x^2+y^2+z^2-1=0$ and $z=0$ is the unit circle in the $x-y$ plane.
If a surface does not have a boundary, we say the boundary is empty.
In this case, we call such a surface closed.\\
Often easiest is to give a local coordinate $u,v$, so $S=\{\underline{x}=\underline{x}(u,v)\}$.
In this case, we define a normal as
$$\frac{\partial\underline{x}}{\partial u}\times\frac{\partial\underline{x}}{\partial v}\left/\middle\|\frac{\partial\underline{x}}{\partial u}\times\frac{\partial\underline{x}}{\partial v}\right\|$$
assuming it is well-defined.
For regular surfaces, this is always well-defined.
It can define the normal consistenly (in terms of sign) or smoothly if the surface is orientable (which we may not define rigourously, sadly).
If the surface is indeed orientable, we use the convention for the orientation of the boundary curve that when we moving along the boundaries, normal vectors are on our left.
\begin{example}
    Consider the hemisphere again using spherical polars $S=\{(\cos\phi\sin\theta,\sin\phi\sin\theta,\cos\theta):\theta\in[0,\pi/2],\phi\in[0,2\pi]\}$
    By calculation we get exactly the vector $\underline{e_r}$ as the normal.
\end{example}
To calculate the area of a surface, we want to partition the surface by a rectangularization of the $u-v$ plane.
So the area of the piece that might look like a parallelogram when we zoom in have an approximated area of $\delta u\delta v\|(\partial \underline{x}/\partial u)\times(\partial \underline{x}/\partial v)\|$, so the area is (from intuition):
\begin{definition}
    The area of the surface $S$ is
    $$\int_S\mathrm dS=\int_S|\mathrm d\underline{S}|=\int_S\left\|\frac{\partial\underline{x}}{\partial u}\times\frac{\partial\underline{x}}{\partial v}\right\|\,\mathrm du\,\mathrm dv$$
\end{definition}
Where we write $\mathrm d\underline{S}=\underline{n}\,\mathrm dS$.
\begin{example}
    We (yet again) look at the hemisphere paramterized by $S=\{(R\cos\phi\sin\theta,R\sin\phi\sin\theta,R\cos\theta):\theta\in[0,\pi/2],\phi\in[0,2\pi]\}$, so
    $$\int_S\mathrm dS=\int_S|\mathrm d\underline{S}|=\int_SR^2\sin\theta\,\mathrm d\theta\,\mathrm d\phi=2\pi R^2$$
\end{example}
We want to use similar method to define the flux integral, which is like the amount of fluid passing though the surface $S$ in unit time.
\begin{definition}
    We define the integral of $f:\mathbb R^3\to\mathbb R$ to be
    $$\int_Sf\,\mathrm dS=\iint_Sf(\underline{x}(u,v))\left\|\frac{\partial\underline{x}}{\partial u}\times\frac{\partial\underline{x}}{\partial v}\right\|\,\mathrm du\,\mathrm dv$$
\end{definition}
Suppose $S:\underline{x}=\underline{x}(u,v),S':\underline{\tilde{x}}=\underline{\tilde{x}}(\tilde{u},\tilde{v})$ are two different parameterizations of the same surface $S$, then we have $\underline{x}(u,v)=\underline{\tilde{x}}(\tilde{u}(u,v),\tilde{v}(u,v))$, where we assume that $\tilde{u},\tilde{u}$ are smooth bijections with smooth inverse.
So we have, by calculus,
$$\frac{\partial\underline{x}}{\partial u}\times\frac{\partial\underline{x}}{\partial v}=\frac{\partial\underline{\tilde{x}}}{\partial\tilde{u}}\times\frac{\partial\underline{\tilde{x}}}{\partial\tilde{v}}\frac{\partial (\tilde{u},\tilde{v})}{\partial (u,v)}$$
So
\begin{align*}
    \int_S f\,\mathrm dS&=\iint_Sf(\underline{x}(u,v))\left\|\frac{\partial\underline{x}}{\partial u}\times\frac{\partial\underline{x}}{\partial v}\right\|\,\mathrm du\,\mathrm dv\\
    &=\iint_{S}f(\underline{\tilde{x}}(\tilde{u}(u,v),\tilde{v}(u,v)))\left\|\frac{\partial\underline{\tilde{x}}}{\partial\tilde{u}}\times\frac{\partial\underline{\tilde{x}}}{\partial\tilde{v}}\frac{\partial (\tilde{u},\tilde{v})}{\partial (u,v)}\right\|\,\mathrm du\,\mathrm dv\\
    &=\iint_{S}f(\underline{\tilde{x}}(\tilde{u}(u,v),\tilde{v}(u,v)))\left\|\frac{\partial\underline{\tilde{x}}}{\partial\tilde{u}}\times\frac{\partial\underline{\tilde{x}}}{\partial\tilde{v}}\right\|\left|\frac{\partial (\tilde{u},\tilde{v})}{\partial (u,v)}\right|\,\mathrm du\,\mathrm dv\\
    &=\iint_{S'}f(\underline{\tilde{x}}(\tilde{u},\tilde{v}))\left\|\frac{\partial\underline{\tilde{x}}}{\partial \tilde{u}}\times\frac{\partial\underline{\tilde{x}}}{\partial\tilde{v}}\right\|\,\mathrm d\tilde{u}\,\mathrm d\tilde{v}\\
    &=\int_{S'} f\,\mathrm dS'
\end{align*}
Just as the Fundamental Theorem of Calculus told us that the integration over a derivative depends only on its endpoints, the integral over a surface of some sort of derivative will only depend on the boundary of the surface.
\footnote{In fact, this is true in manifolds of even higher dimensions, which is known as Stokes' Theorem.}
Then, for a vector field $\underline{F}$, we define the flux integral of it over the surface $S$ by
$$\int_S\underline{F}\cdot\mathrm d\underline{S}=\int_S\underline{F}\cdot\underline{n}\,\mathrm dS$$
    \section{Divergence, Curl and Laplacian}
\subsection{Definitions}
\begin{definition}
    For a vector field $\underline{F}:\mathbb R^3\to\mathbb R^3$, we define the divergence
    $$\operatorname{div}\underline{F}=\nabla\cdot\underline{F}=\frac{\partial F_i}{\partial x_i}$$
    where the summation convention applies.
\end{definition}
\begin{definition}
    For a vector field $\underline{F}:\mathbb R^3\to\mathbb R^3$, we define the curl
    $$\operatorname{curl}\underline{F}=\nabla\times\underline{F}=\epsilon_{ijk}\frac{\partial F_k}{\partial x_j}\underline{e_i}$$
    where again the summation convention applies.
\end{definition}
\begin{definition}
    For a function $f:\mathbb R^3\to\mathbb R$, we define the Laplacian
    $$\Delta f=\nabla^2f=\nabla\cdot(\nabla f)=\frac{\partial^2 f}{\partial x_i\partial x_i}$$
    where yet again we have the summation convention.
\end{definition}
\begin{note}
    All these are in Cartesians.
\end{note}
\begin{example}
    Consider the vector field $\underline{F}(\underline{x})=\underline{x}$, then $\nabla\cdot\underline{F}=3$.
    Also $(\nabla\times\underline{F})_i=\epsilon_{ijk}\frac{\partial F_k}{\partial x_j}=\epsilon_{ijk}\delta_{jk}=0$, hence $\nabla\times\underline{F}=\underline{0}$.
\end{example}
\begin{proposition}
    We have the following identities:
    $$\nabla(fg)=(\nabla f)g+f(\nabla g)$$
    $$\nabla\cdot(f\underline{F})=(\nabla f)\cdot\underline{F}+f(\nabla\cdot\underline{F})$$
    $$\nabla\times (fF)=(\nabla f)\times\underline{F}+f(\nabla\times\underline{F})$$
    $$\nabla(\underline{F}\cdot\underline{G})=\underline{F}\times(\nabla\times\underline{G})+\underline{G}\times(\nabla\times\underline{F})+(\underline{F}\cdot\nabla)\underline{G}+(\underline{G}\cdot\nabla)\underline{F}$$
    $$\nabla\times(\underline{F}\times\underline{G})=\underline{F}(\nabla\cdot\underline{G})-\underline{G}(\nabla\cdot\underline{F})+(\underline{G}\cdot\nabla)\underline{F}-(\underline{F}\cdot\nabla)\underline{G}$$
    $$\nabla\cdot(\underline{F}\times\underline{G})=(\nabla\times\underline{F})\cdot\underline{G}-\underline{F}\cdot(\nabla\times\underline{G})$$
\end{proposition}
\begin{proof}
    Trivial.
\end{proof}
Of course we want, can can compute these three quantities in curvilinear coordinates, but we cannot do it directly since the basis vectors are not constant.
However we can expand everything and get
\begin{proposition}
    For a vector field $\underline{F}$ under a curvilinear coordinate $\underline{F}=F_u\underline{e_u}+F_v\underline{e_v}+F_w\underline{e_w}$,
    $$\nabla\cdot\underline{F}=\frac{1}{h_uh_vh_w}\sum_{u,v,w}^{\rm cyc}\frac{\partial}{\partial u}(h_vh_wF_u)$$
    $$\nabla\times\underline{F}=\sum_{u,v,w}^{\rm cyc}\frac{1}{h_vh_w}\left( \frac{\partial}{\partial v}(h_wF_w)-\frac{\partial}{\partial w}(h_vF_v) \right)\underline{e_u}$$
    And for a scalar function $f$,
    $$\nabla^2f=\frac{1}{h_uh_vh_w}\sum_{u,v,w}^{\rm cyc}\frac{\partial}{\partial u}\left(\frac{h_uh_w}{h_u}\frac{\partial f}{\partial u}\right)$$
\end{proposition}
\begin{proof}
    Trivial calculations.
\end{proof}
If one is bored, one can try and find the formulas for cylindral and spherical coordinates:
$$\nabla^2f=\frac{1}{\rho}\frac{\partial}{\partial\rho}\left( \rho\frac{\partial f}{\partial\rho} \right)+\frac{1}{\rho^2}\frac{\partial^2f}{\partial \phi^2}+\frac{\partial^2f}{\partial z^2}$$
$$\nabla^2f=\frac{1}{r^2}\frac{\partial}{\partial r}\left( r^2\frac{\partial f}{\partial r} \right)+\frac{1}{r^2\sin\theta}\frac{\partial}{\partial\theta}\left( \sin\theta\frac{\partial f}{\partial\theta} \right)+\frac{1}{r^2\sin\theta}\frac{\partial^2f}{\partial\phi^2}$$
The reason why we need these notions is for the generalization of fundamental Theorem of Calculus to general integrals, where some of these operators will be used as a substituent of derivative.\\
The reason we need Laplacians is that the PDE $\nabla^2f=0$, whose solutions are called harmonic functions, is pretty important.
One of their properties that once they are twice differentiable (so as to let the equation make sense), then they are infinitely differentiable.
Even better, they are all analytic, i.e. can be expressed in terms of power series.
\subsection{Relationships between the Operators}
\begin{proposition}
    Let $f:\mathbb R^3\to\mathbb R$ and $\underline{F}:\mathbb R^3\to\mathbb R^3$, then $\nabla\times\nabla f=0$ and $\nabla \cdot(\nabla\times\underline{F})=0$.
\end{proposition}
\begin{proof}
    Trivial.
\end{proof}
Hence if $\underline{F}$ is conservative, then it has zero curl.
The reverse implication is true when the domain is simply connected.
For example, if we take $\mathbb R^3\setminus\{(0,0,z):z\in\mathbb R\}$ as our domain, then this is not simply connected, but $\mathbb R^3\setminus\{(0,0,0)\}$ is.\\
If there exists vector fields $\underline{A}$ such that $\underline{F}=\nabla\times\underline{A}$, we say $\underline{A}$ is a vector potential of $\underline{F}$.
So if $\nabla\cdot\underline{F}=0$, we say $\underline{F}$ is solenoidal.
The existence of a vector potential for $\underline{F}$ implies $\underline{F}$ is solenoidal.
The reverse implication is true when the domain is $2$-connected, that is, it is simply connected and the second homotopy group is trivial.
For example, $\mathbb R^3$ is $2$-connected but $\mathbb R^3\setminus\{(0,0,0)\}$ is not.
    \subsection{Bonus: Topology via Calculus and de Rham Cohomology}
If $\underline{F}$ is a vector field with simply connected domain, we know that $\nabla\times\underline{F}=0$ implies that $\underline{F}$ is conservative.
We can conversely use it to show that certain domain is not simply connected.
\begin{example}
    Suppose $\mathbb R\setminus\{(0,0,z):z\in\mathbb R\}$ is simply connected.
    Then consider the vector field
    $$\underline{F}=\frac{1}{x^2+y^2}(-y,x,0)$$
    So it is well defined and smooth on the said domain.
    It also has zero curl.
    So exists scalar function $f$ such that $\underline{F}=\nabla f$, hence the line integral of it along any loop is zero.
    However, let us consider the curve in the $x-y$ plane:
    $$C:[0,2\pi]\ni t\mapsto\begin{pmatrix}
        \cos t\\
        \sin t\\
        0
    \end{pmatrix}$$
    Then
    $$\int_C\underline{F}\cdot\mathrm d\underline{x}=2\pi\neq 0$$
    Contradiction.
\end{example}
    \section{Integral Theorems}
\subsection{Green's Theorem}
\begin{proposition}[Green's Theorem]
    For continuously differentiable functions $P=P(x,y),Q=Q(x,y)$ and a bounded region $A\subset\mathbb R^2$ with piecewise smooth boundary $\partial A$, we have
    $$\oint_{\partial A}P\,\mathrm dx+Q\,\mathrm dy=\iint_A\left( \frac{\partial Q}{\partial x}-\frac{\partial P}{\partial y} \right)\,\mathrm dx\,\mathrm dy$$
    where the direction of $\partial A$ is taken such that the region on the left of motion.
\end{proposition}
Note that the choice of direction is consistent with the convention we used for surfaces in $\mathbb R^3$ if we consider the normal to be pointing out of paper.
We shall prove the case where $A$ is rectangular, i.e. $A=\{(x,y):x\in [a,b],y\in[c,d]\}$.
\begin{proof}[Proof of Rectangular Case]
    \begin{align*}
        \iint_A\left( \frac{\partial Q}{\partial x}-\frac{\partial P}{\partial y} \right)\,\mathrm dx\,\mathrm dy
        &=\int_c^d\int_a^b\frac{\partial Q}{\partial x}\,\mathrm dx\,\mathrm dy-\int_a^b\int_c^d\frac{\partial P}{\partial y}\,\mathrm dy\,\mathrm dx\\
        &=\int_c^dQ(b,y)-Q(a,y)\,\mathrm dy-\int_a^bP(x,d)-P(x,c)\,\mathrm dx\\
        &=\oint_{\partial A}P\,\mathrm dx+Q\,\mathrm dy
    \end{align*}
    As desired.
\end{proof}
The general case can be thought of gluing many rectangles together.
\begin{example}
    Suppose $Q=x/2,P=-y/2$, then
    $$\oint_{\partial A}P\,\mathrm dx+Q\,\mathrm dy=\iint_A\,\mathrm dx\,\mathrm dy=\operatorname{Area}(A)$$
    Let $A$ be the ellipse $x^2/a^2+y^2/b^2\le1$, which by integrating the line integral on the left, we get $\operatorname{Area}(A)=\pi ab$.
\end{example}
\subsection{Stokes' Theorem}
\begin{proposition}
    For a continuously differentiable vector field $\underline{F}$ and any orientable surface $S$ with piecewise smooth boundary, then
    $$\int_S\nabla\times \underline{F}\cdot\mathrm d\underline{S}=\oint_{\partial S}\underline{F}\cdot\mathrm d\underline{x}$$
\end{proposition}
The orientability is important since we will need a consistent choice of normal on $S$ that varies smoothly from point to point.
So surfaces can be said to have two sides, the inside and outside.
An example of a non-orientable surface is the Mobius strip.
\begin{example}
    Consider a spherical cap
    $$S=\{\underline{x}=(\cos\phi\sin\theta,\sin\phi\sin\theta,\cos\theta)^\top:\phi\in[0,2\pi],\theta\in[0,\alpha]\}$$
    Let $F(\underline{x})=(-x^2y,0,0)^\top$, so $\nabla\times\underline{F}=(0,0,x^2)^\top$.
    Now $\mathrm d\underline{S}=\underline{e_r}\sin\theta\,\mathrm d\theta\mathrm d\phi$.
    So
    $$\int_S\nabla\times\underline{F}\cdot\mathrm d\underline{S}=\int_0^\alpha\int_0^{2\pi}(\cos\phi\sin\theta)^2\cos\theta\sin\theta\,\mathrm d\phi\mathrm d\theta=\frac{\pi}{4}\sin^4\alpha$$
    Now $\partial S:[0,2\pi]\ni t\mapsto (\cos t\sin\alpha,\sin t\sin\alpha,\cos\alpha)^\top$, we can calculate to find
    $$\oint_{\partial S}\underline{F}\cdot\mathrm d\underline{x}=\frac{\pi}{4}\sin^4\alpha$$
    Which is equal to the original value.
\end{example}
\begin{example}
    If $S$ is a closed surface, then its boundary is $0$, hence by Stokes' Theorem,
    $$\int_S\nabla \times\underline{F}\cdot\mathrm d\underline{S}=0$$
    which just looks like what we get when we integrate a closed loop.
\end{example}
\begin{proposition}
    If $\underline{F}$ is continuously differentiable and
    $$\oint_C\underline{F}\cdot\mathrm d\underline{x}=0$$
    for any closed loop $C$, then $\nabla F=\underline{0}$.
\end{proposition}
Hence zero circulation implies irrotaion.
\begin{proof}
    Suppose $\underline{F}$ satisfies all conditions but $\nabla\times\underline{F}\neq \underline{0}$, then there is a unit vector $\underline{k}$ such that it is nonzero in the $\underline{k}$ direction, then if there is some $\epsilon>0$ such that $\underline{k}\cdot(\nabla\times\underline{F}(\underline{x_0}))>\epsilon$, then there is some $\delta>0$ such that $|\underline{x}-\underline{x_0}|<\delta$ implies $\underline{k}\cdot(\nabla\times\underline{F})>\epsilon/2>0$.\\
    Now consider the ball $|\underline{x}-\underline{x_0}|<\delta$ and we choose a disk $D$ inside it, we have
    $$0=\left|\oint_{\partial D}\underline{F}\,\mathrm d\underline{x}\right|=\left|\int_D\nabla\times\underline{F}\cdot\mathrm d\underline{S}\right|\ge\frac{\epsilon}{2}\operatorname{Area}(D)>0$$
    Contradiction.
\end{proof}
\begin{example}
    Let $S_{\epsilon}$ be any sufficiently nice surface contained inside a disk with radius $\epsilon>0$ centered at $\underline{x}=\underline{x_0}$ with normal $\underline{k}$.
    If
    \begin{align*}
        \int_{S_\epsilon}\nabla\times\underline{F}\cdot\mathrm d\underline{S}&=\int_{S_\epsilon}\nabla\times\underline{F}(\underline{x_0})\cdot\mathrm d\underline{S}+\left( \int_{S_\epsilon}\nabla\times(\underline{F}-\underline{F}(\underline{x_0}))\cdot\mathrm d\underline{S} \right)
        \\
        &=\underline{k}\cdot\nabla\times\underline{F}(\underline{x_0})\operatorname{Area}(S_\epsilon)+\int_{S_\epsilon}\nabla\times(\underline{F}-\underline{F}(\underline{x_0}))\cdot\mathrm d\underline{S}
    \end{align*}
    Now we claim that the last term is $o(\operatorname{Area}(S_\epsilon))$ as $\epsilon\to 0$.
    Indeed,
    \begin{align*}
        \left| \int_{S_\epsilon}\nabla\times(\underline{F}-\underline{F}(\underline{x_0}))\cdot\mathrm d\underline{S} \right|
        &\le\int_{S_\epsilon}|\nabla\times(\underline{F}-\underline{F}(\underline{x_0}))|\cdot\mathrm d\underline{S}\\
        &\le\sup_{\underline{x}\in S_\epsilon}|\nabla\times(\underline{F}(\underline{x})-\underline{F}(\underline{x_0}))|\operatorname{Area}(S_\epsilon)\\
        &=o(\operatorname{Area}(S_\epsilon))
    \end{align*}
    As $\underline{F}$ is continuously differentiable.
    Therefore by Stokes' Theorem,
    \begin{align*}
        \frac{1}{\operatorname{Area}(S_\epsilon)}\oint_{\partial S_\epsilon}\underline{F}\cdot\mathrm d\underline{x}
        &=\frac{1}{\operatorname{Area}(S_\epsilon)}\int_{S_\epsilon}\nabla\times\underline{F}\cdot\mathrm d\underline{S}\\
        &=\underline{k}\cdot\nabla\times\underline{F}(\underline{x_0})+o(1)
    \end{align*}
    As $\epsilon\to 0$.
    So the curl is the infinitesimal circulation around the normal $\underline{k}$ per unit area.
\end{example}
    \subsection{Bonus: Mobius Band and Stokes'}
We all know what a Mobius Band (or Mobius Strip) is.
In particular, it is not orientable since it only have one side.
Consider $\underline{F}=(-y,x,0)^\top/(x^2+y^2)$, so $\nabla\times\underline{F}=\underline{0}$ whenever $x^2+y^2>0$.
If we are given the parameterization
$$S=\left\{\underline{x}(u,v)=\begin{pmatrix}
    (1+\frac{v}{2}\cos\frac{u}{2})\cos u\\
    (1+\frac{v}{2}\cos\frac{u}{2})\sin u\\
    \frac{v}{2}\sin\frac{u}{2}
\end{pmatrix}:u\in[0,2\pi],v\in[-1,1]\right\}$$
Then if we apply Stokes' Theorem (which we should not), then
$$0=\int_S\nabla\times\underline{F}\cdot\mathrm d\underline{S}=\oint_{\partial S}\underline{F}\cdot\mathrm d\underline{x}$$
But the boundary, which is parameterized as
$$[0,4\pi]\ni t\mapsto \begin{pmatrix}
    (1+\frac{1}{2}\cos\frac{t}{2})\cos t\\
    (1+\frac{1}{2}\cos\frac{t}{2})\sin t\\
    \frac{1}{2}\sin\frac{t}{2}
\end{pmatrix}$$
Then we have
$$\oint_{\partial S}\underline{F}\cdot\mathrm d\underline{x}=4\pi\neq 0$$
Contradiction.
    \subsection{Divergence Theorem}
\begin{proposition}
    Let $\underline{F}$ be a continuously differentiable vector field, and let $V$ be a volume in $\mathbb R^3$ with piecewise regular boundary $\partial V$, then
    $$\int_V\nabla\cdot\underline{F}\,\mathrm dV=\int_{\partial V}\underline{F}\cdot\mathrm d\underline{S}$$
    where the normal points out of the volume $V$.
\end{proposition}
\begin{proposition}
    Let $\underline{F}$ be a continuously differentiable vector field in $\mathbb R^2$, and let $D$ be a subset of $\mathbb R^2$ be a region with piecewise smooth boundary $\partial D$, then
    $$\int_D\nabla\cdot\underline{F}\,\mathrm dA=\int_{\partial D}\underline{F}\cdot\underline{n}\,\mathrm ds$$
    where $\underline{n}$ points out of the region $D$.
\end{proposition}
\begin{example}
    Let $\underline{F}(\underline{x})=\underline{x}$ and $V$ the cylinder, so
    $$V=\{\underline{x}=\underline{x}(\rho,\phi,z):0\le\rho\le R,0\le\phi\le 2\pi,-h\le z\le h\}$$
    So $\nabla\cdot\underline{F}=3$, hence
    $$\int_V\nabla\cdot\underline{F}\,\mathrm dV=3\int_V\mathrm dV=6\pi R^2h$$
    As for the surface integral, we write $\partial V=S_+\cup S_-\cup S$ where $S_+,S_-$ are the top and lower disks, and $S$ is the curved surface in between.
    $$S=\{R\underline{e_\rho}+z\underline{e_z}:z\in[-h,h],\phi\in [0,2\pi]\}$$
    So $\mathrm d\underline{S}=\underline{e_\rho}R\,\mathrm d\phi\,\mathrm dz$, hence by calculation,
    $$\int_S\underline{F}\cdot\mathrm d\underline{S}=4\pi R^2h$$
    Now $S_{\pm}=\{\rho\underline{e_\rho}\pm h\underline{e_z}:\rho\in[0,R],\phi\in[0,2\pi]\}$.
    We will also find that $\mathrm d\underline{S_\pm}=\pm\underline{e_z}\rho\,\mathrm d\rho\,\mathrm d\phi$.
    $$\int_{S_{\pm}}\underline{F}\cdot\mathrm d\underline{S_{\pm}}=\pi R^2h$$
    So adding them together does give $6\pi R^2h$.
\end{example}
\begin{proposition}
    If $\underline{F}$ is continuously differentiable and for all closed surfaces $S$ we have
    $$\int_S\underline{F}\cdot\mathrm d\underline{S}=0$$
    Then $\nabla\cdot\underline{F}=0$.
\end{proposition}
\begin{proof}
    Assume that it is not zero, so WLOG we can take a point $\underline{x_0}$ such that $\nabla\cdot\underline{F}(\underline{x_0})=\epsilon>0$, then there is some $\delta>0$ with $|\underline{x}-\underline{x_0}|<\delta\implies\nabla\cdot\underline{F}(\underline{x})>\epsilon/2$.
    Take the volume $V$ to be the ball $\{\underline{x}\in\mathbb R^3:|\underline{x}-\underline{x_0}|<\delta\}$ with boundary $\partial V$, then
    $$0=\int_{\partial V}\underline{F}\cdot\mathrm d\underline{S}=\int_V\nabla\cdot\underline{F}\,\mathrm dV>\frac{\epsilon}{2}\operatorname{Volume}(V)>0$$
    by Divergence Theorem.
    Contradiction.
\end{proof}
\begin{example}
    Let $V_\epsilon$ be a volume contained inside a ball of radius $\epsilon$ centered at $\underline{x_0}$.
    Then
    \begin{align*}
        \int_{\partial V_\epsilon}\underline{F}\cdot\mathrm d\underline{S}&=\int_{V_\epsilon}\nabla\cdot\underline{F}\,\mathrm dV\\
        &=\int_{V_\epsilon}\nabla\cdot\underline{F}(\underline{x_0})\,\mathrm dV+\left( \int_{V_\epsilon}\nabla\cdot\underline{F}\,\mathrm dV-\int_{V_\epsilon}\nabla\cdot\underline{F}(\underline{x_0})\,\mathrm dV \right)\\
        &=\nabla\cdot\underline{F}\operatorname{Volume}(V_\epsilon)+\left( \int_{V_\epsilon}\nabla\cdot\underline{F}-\nabla\cdot\underline{F}(\underline{x_0})\,\mathrm dV \right)
    \end{align*}
    But as we did before,
    \begin{align*}
        \left| \int_{V_\epsilon}\nabla\cdot\underline{F}-\nabla\cdot\underline{F}(\underline{x_0})\,\mathrm dV \right|&\le\operatorname{Volume}(V_\epsilon)\sup_{\underline{x}\in V_\epsilon}|\nabla\cdot\underline{F}(\underline{x})-\nabla\cdot\underline{F}(\underline{x_0})|\\
        &=o(\operatorname{Volume(V_\epsilon)})
    \end{align*}
    as $\epsilon\to 0^+$.
    Hence
    $$\nabla\cdot\underline{F}(\underline{x_0})=\lim_{\epsilon\to0^+}\frac{1}{\operatorname{Volume}(V_\epsilon)}\int_{\partial V_\epsilon}\underline{F}\cdot\mathrm d\underline{S}$$
    That is, $\nabla\cdot\underline{F}$ measures the infinitesimal flux per unit volume.
    Therefore $\nabla\cdot\underline{F}(\underline{x_0})>0$ means that the field is going out of $\underline{x_0}$, and it being negative means that the field is going into $\underline{x_0}$.
    If it is zero at that point, then the field is incompressible there.
\end{example}
\begin{example}
    1. Take again $\underline{F}(\underline{x})=\underline{x}$ and $V_\epsilon=\{\underline{x}:|\underline{x}|<\epsilon\}$, then we calculate
    \begin{align*}
        \nabla\cdot\underline{F}(\underline{0})&=\lim_{\epsilon\to 0^+}\frac{1}{\operatorname{Volume}(V_\epsilon)}\int_{\partial V_\epsilon}\underline{F}\cdot\mathrm d\underline{S}\\
        &=3
    \end{align*}
    as desired.\\
    2. Call equations of the form
    $$\frac{\partial\rho}{\partial t}+\nabla\cdot\underline{J}=0$$
    as \textit{conservation laws}.
    We claim that if $|\underline{J}|\to 0$ as $|\underline{x}|\to\infty$, then the charge
    $$Q(t)=\int_{\mathbb R^3}\rho(\underline{x},t)\,\mathrm dV$$
    remains constant.
    We differentiate to get
    \begin{align*}
        \frac{\mathrm dQ}{\mathrm dt}&=\int_{\mathbb R^3}\frac{\partial\rho}{\partial t}\,\mathrm dV\\
        &=-\int_{\mathbb R^3}\nabla\cdot\underline{J}\,\mathrm dV\\
        &=-\lim_{R\to\infty}\int_{|\underline{x}|<R}\nabla\cdot\underline{J}\,\mathrm dV\\
        &=-\lim_{R\to\infty}\int_{|\underline{x}|=R}\underline{J}\cdot\mathrm d\underline{S}\\
        &=0
    \end{align*}
    So $Q$ is constant.
\end{example}
    \subsection{Bonus: Noether's Theorem}
If a system of differential equations admits a (translational, rotational, etc.) symmetry, then something is conserved.
It is a very pure theorem but has significant physical meanings.
So the conservation laws in mathematical physics arises naturally from the time independence, spacial independence or directional independence of a physical law.
People who are interested can consult the book \textit{Applications of Lie Groups to Differential Equations} by Peter Oliver.
    \subsection{Sketch Proofs}
\begin{proof}[Proof of Divergence Theorem in Convex Domains]
    Consider $\underline{F}=F_z\underline{e_z}$ and a volume $V$ with $\partial V=S_+\cup S_-$ divided by a surface such that both $S_+,S_-$ project to the same surface $A$ on the $x-y$ plane (which is possible since the domain is convex).
    We describe the surfaces as
    $$S_{\pm}=\{\underline{x}=\underline{x}(x,y)=\begin{pmatrix}
        x\\
        y\\
        g_{\pm}(x,y)
    \end{pmatrix}:(x,y)\in A\}$$
    Therefore
    \begin{align*}
        \int_V\frac{\partial F_z}{\partial z}\,\mathrm dV&=\iint_A\left( \int_{g_-(x,y)}^{g_+(x,y)} \frac{\partial F_z}{\partial z}\,\mathrm dz\right)\,\mathrm dx\,\mathrm dy\\
        &=\iint_A(F_z(x,y,g_+(x,y))-F_z(x,y,g_-(x,y)))\,\mathrm dx\,\mathrm dy
    \end{align*}
    Now note that
    $$\mathrm d\underline{S}=\frac{\partial\underline{x}}{\partial x}\times\frac{\partial\underline{x}}{\partial y}\,\mathrm dx\,\mathrm dy=\begin{pmatrix}
        -\partial g_\pm/\partial x\\
        -\partial g_\pm/\partial y\\
        1
    \end{pmatrix}\,\mathrm dx\,\mathrm dy$$
    Now we need the normal to point out of $V$, hence on $S_\pm$.
    $$\mathrm d\underline{S}=\pm\begin{pmatrix}
        -\partial g_\pm/\partial x\\
        -\partial g_\pm/\partial y\\
        1
    \end{pmatrix}\,\mathrm dx\,\mathrm dy$$
    So
    $$\int_{\partial V}F_z\underline{e_z}\cdot\mathrm d\underline{S}=\iint_A(F_z(x,y,g_+(x,y))-F_z(x,y,g_-(x,y)))\,\mathrm dx\,\mathrm dy=\int_V\frac{\partial F_z}{\partial z}\,\mathrm dV$$
    Now we can do the same thing on $F_y\underline{e_y}$ and $F_x\underline{e_x}$, so adding them up gives the theorem by linearity.
\end{proof}
Note that the same proof works for two dimensional case.
Now we want to proof Green's Theorem by the divergence theroem in two dimensions.
After that we shall prove that Green's Theorem implies Stokes' Theorem.
\begin{proof}[Proof of Green's using Divergence]
    Consider the vector field $\underline{F}=(Q,-P)$, so by Divergence Theorem,
    \begin{align*}
        \iint_A\left( \frac{\partial Q}{\partial x}-\frac{\partial P}{\partial y} \right)\,\mathrm dx\,\mathrm dy
        &=\int_A\nabla\cdot\underline{F}\,\mathrm dA\\
        &=\oint_{\partial A}\underline{F}\cdot\underline{n}\,\mathrm ds\\
        &=\oint_{\partial A}\begin{pmatrix}
            Q\\
            -P
        \end{pmatrix}\cdot\begin{pmatrix}
            y^\prime(s)\\
            -x^\prime(s)
        \end{pmatrix}\,\mathrm ds\\
        &=\oint_{\partial A}P\,\mathrm dx+Q\,\mathrm dy
    \end{align*}
    which is precisely Green's.
\end{proof}
\begin{proof}[Proof of Stokes' by Green's]
    For $A\subset\mathbb R^2,Q=Q(u,v),P=P(u,v)$, we have
    $$\iint_A\left( \frac{\partial Q}{\partial u}-\frac{\partial P}{\partial v} \right)\,\mathrm du\,\mathrm dv=\oint_{\partial A}P\,\mathrm du+Q\,\mathrm dv$$
    Consider the surface $S=\{\underline{x}(u,v):(u,v)\in A\}$ and boundary $\partial S=\{\underline{x}(u,v):(u,v)\in\partial A\}$.
    Choose
    $$\begin{cases}
        P=\underline{F}(\underline{x}(u,v))\cdot\partial\underline{x}/\partial u\\
        Q=\underline{F}(\underline{x}(u,v))\cdot\partial\underline{x}/\partial v
    \end{cases}$$
    Note that
    \begin{align*}
        P\,\mathrm du+Q\,\mathrm dv
        &=\underline{F}(\underline{x}(u,v))\cdot\frac{\partial\underline{x}}{\partial u}\,\mathrm du+\underline{F}(\underline{x}(u,v))\cdot\frac{\partial\underline{x}}{\partial v}\,\mathrm dv\\
        &=\underline{F}(\underline{x}(u,v))\cdot\left( \frac{\partial\underline{x}}{\partial u}\,\mathrm du+\frac{\partial\underline{x}}{\partial v}\,\mathrm dv \right)\\
        &=\underline{F}(\underline{x}(u,v))\cdot\mathrm d\underline{x}(u,v)\\
        \implies \oint_{\partial A}P\,\mathrm du+Q\,\mathrm dv&=\oint_{\partial S}\underline{F}\cdot\mathrm d\underline{x}
    \end{align*}
    On the other hand, we can differentiate to get
    \begin{align*}
        \frac{\partial Q}{\partial u}&=\frac{\partial}{\partial u}\left( \underline{F}(\underline{x}(u,v))\cdot\partial\underline{x}/\partial v \right)\\
        &=\frac{\partial x_j}{\partial u}\frac{\partial F_i}{\partial x_j}\frac{\partial x_i}{\partial v}+F_i(\underline{x}(u,v))\frac{\partial^2 x_i}{\partial u\partial v}
    \end{align*}
    Similarly
    $$\frac{\partial P}{\partial v}=\frac{\partial x_j}{\partial v}\frac{\partial F_i}{\partial x_j}\frac{\partial x_i}{\partial u}+F_i(\underline{x}(u,v))\frac{\partial^2 x_i}{\partial v\partial u}$$
    Now combining these two and using the symmetries in second partial derivatives,
    \begin{align*}
        \left(\frac{\partial Q}{\partial u}-\frac{\partial P}{\partial v}\right)\,\mathrm du\,\mathrm dv&=\left(\frac{\partial x_j}{\partial u}\frac{\partial F_i}{\partial x_j}\frac{\partial x_i}{\partial v}-\frac{\partial x_j}{\partial v}\frac{\partial F_i}{\partial x_j}\frac{\partial x_i}{\partial u}\right)\,\mathrm du\,\mathrm dv\\
        &=(\nabla\times\underline{F})\cdot\left( \frac{\partial\underline{x}}{\partial u}\times\frac{\partial\underline{x}}{\partial v} \right)\,\mathrm du\,\mathrm dv\\
        &=(\nabla\times\underline{F})\cdot\mathrm d\underline{S}
    \end{align*}
    Combining this with Green's Theorem gives the result.
\end{proof}
    \section{Maxwell's Equations}
Just a little identity on Laplacian of vector fields
\begin{proposition}
    We have
    $$\nabla^2\underline{F}=\nabla(\nabla\cdot\underline{F})-\nabla\times(\nabla\times\underline{F})$$
    where $(\nabla^2\underline{F})_i=\nabla^2\underline{F_i}$
\end{proposition}
\subsection{Introduction to Electromagnetism}
We have the electric field $\underline{E}=\underline{E}(\underline{x},t)$, the magnetic field $\underline{B}=\underline{B}(\underline{x},t)$, charge density $\rho=\rho(\underline{x},t)$ and current density $\underline{J}=\underline{J}(\underline{x},t)$.
The Maxwell's Equations state that
$$\begin{cases}
    \nabla\cdot\underline{E}=\epsilon_0^{-1}\rho\\
    \nabla\cdot\underline{B}=0\\
    \nabla\times\underline{E}+\partial\underline{B}/\partial t=0\\
    \nabla\times\underline{B}-\mu_0\epsilon_0\partial\underline{E}/\partial t=\mu_0\underline{J}
\end{cases}$$
where $\epsilon_0$ is the permitivity and $\mu_0$ is the permeability of free space with $\mu_0\epsilon_0=c^{-2}$ where $c$ is the speed of light.
Take the divergence of the fourth equation gives
$$0=\mu_0\epsilon_0\frac{\partial}{\partial t}(\nabla\cdot\underline{E})+\mu_0\nabla\cdot\underline{J}\implies 0=\frac{\partial\rho}{\partial t}+\nabla\cdot\underline{J}$$
\subsection{Integral Forms of Maxwell's Equations}
If we integrate then first equation and use divergence theorem to integrate the electric field over the flux of the boundary of a volume, we get
$$\int_{\partial V}\underline{E}\cdot\mathrm d\underline{S}=\int_V\nabla\cdot\underline{E}\,\mathrm dV=\frac{1}{\epsilon_0}\int_V\rho\,\mathrm dV=\frac{Q}{\epsilon_0}$$
where $Q$ is the total charge of the volume $V$.
Do exactly the same thing with th second equation then gives
$$\int_{\partial V}\underline{B}\cdot\mathrm d\underline{S}=\int_V\nabla\cdot\underline{B}\,\mathrm dV=0$$
which is saying that there is no magnetic monopoles since we cannot have a singular pole that emit magnetic field out of a volume.
In fact, if somewhere there exists a magnetic monopole, then charges are necessarily quantized.\\
As for the third equation, we have
$$\oint_{\partial S}\underline{E}\cdot\mathrm d\underline{x}=\int_S\nabla\times\underline{E}\cdot\mathrm d\underline{S}=-\frac{\mathrm d}{\mathrm dt}\int_S\underline{B}\,\mathrm d\underline{S}$$
So change in magnetic flux induces electric field.
Similarly, in the fourth equation,
$$\oint_{\partial S}\underline{B}\cdot\mathrm d\underline{x}=\int_S\nabla\times\underline{B}\cdot\mathrm d\underline{S}=\mu_0\epsilon_0\frac{\mathrm d}{\mathrm dt}\int_S\underline{E}\cdot\mathrm d\underline{S}+\mu_0\int_S\underline{J}\cdot\mathrm d\underline{S}$$
\subsection{Electromagnetic Waves}
In a free space, where $\rho=\underline{J}=\underline{0}$, then
\begin{align*}
    \nabla^2\underline{E}&=\nabla(\nabla\cdot\underline{E})-\nabla\times(\nabla\times\underline{E})\\
    &=0-\nabla\times\left(-\frac{\partial\underline{B}}{\partial t}\right)\\
    &=\frac{\partial}{\partial t}(\nabla\times\underline{B})\\
    &=\mu_0\epsilon_0\frac{\partial^2\underline{E}}{\partial t^2}\\
    \implies 0&=\nabla^2\underline{E}-\frac{1}{c^2}\frac{\partial^2\underline{E}}{\partial t^2}
\end{align*}
which is the wave equation of a wave with speed $c$.
If we do the same thing to $\underline{B}$, we obtain
\begin{align*}
    \nabla^2\underline{B}&=\nabla(\nabla\cdot\underline{B})-\nabla\times(\nabla\times\underline{B})\\
    &=0-\nabla\times\left(\frac{1}{c^2}\frac{\partial\underline{E}}{\partial t}\right)\\
    &=-\frac{1}{c^2}\frac{\partial}{\partial t}(\nabla\times\underline{E})\\
    &=\frac{1}{c^2}\frac{\partial^2\underline{B}}{\partial t^2}\\
    \implies 0&=\nabla^2\underline{B}-\frac{1}{c^2}\frac{\partial^2\underline{B}}{\partial t^2}
\end{align*}
\subsection{Electrostatics and Magnitostatics}
Assume that everything is time-independent, then $t$-derivatives are all $0$, which produces
$$\begin{cases}
    \nabla\cdot\underline{E}=\epsilon_0^{-1}\rho\\
    \nabla\cdot\underline{B}=0\\
    \nabla\times\underline{E}=0\\
    \nabla\times\underline{B}=\mu_0\underline{J}
\end{cases}$$
So if we work in $\mathbb R^3$ which is $2$-connected, we can write $\underline{E}=-\nabla\phi$ and $\underline{B}=\nabla\times\underline{A}$ for some $\phi,\underline{A}$.
$\phi$ is called the electric potential and $\underline{A}$ the magnetic potential.
So Maxwell's equations reduce to
$$\begin{cases}
    -\nabla^2\phi=\rho/\epsilon_0\\
    \nabla\times(\nabla\times\underline{A})=\mu_0\underline{J}
\end{cases}$$
The first one is called the Poisson's Equation.
\subsection{Gauge Invariance}
In a $2$-connected domain, we can always write $\underline{B}=\nabla\times\underline{A}$.
But note that the equation still holds by adding the gradient of some scalar function to $\underline{A}$, so $\underline{B}$ is invariant under $\underline{A}\mapsto\underline{A}+\nabla\chi$ for $\chi=\chi(\underline{x},t)$.
If we put the vector potential into the third equation,
$$\nabla\times\left( \underline{E}+\frac{\partial\underline{B}}{\partial t} \right)=0$$
So we can write $\underline{E}=-\nabla\phi-\partial\underline{B}/\partial t$, so we have
$$-\nabla^2\phi-\frac{\partial}{\partial t}(\nabla\cdot\underline{A})=\frac{\rho}{\epsilon_0}$$
And
$$\nabla\times(\nabla\times A)+\frac{1}{c^2}\nabla\left( \frac{\partial\phi}{\partial t} \right)+\frac{1}{c^2}\frac{\partial^2\underline{A}}{\partial t^2}=\mu_0\underline{J}$$
But by a known identity on curl of curl,
$$-\nabla^2\underline{A}+\frac{1}{c^2}\frac{\partial^2\underline{A}}{\partial t^2}+\nabla\left( \frac{1}{c^2}\frac{\partial\phi}{\partial t}+\nabla\cdot\underline{A} \right)=\mu_0\underline{J}$$
We now choose the scalar field $\chi$ such that
$$\frac{1}{c^2}\frac{\partial\phi}{\partial t}+\nabla\cdot\underline{A}=0$$
under $\underline{A}\mapsto\underline{A}+\nabla\chi$.
So we can get an equation similar to a wave equation
$$-\nabla^2\underline{A}+\frac{1}{c^2}\frac{\partial^2\underline{A}}{\partial t^2}=\mu_0\underline{J}$$
and
$$-\nabla^2\phi+\frac{1}{c^2}\frac{\partial^2\phi}{\partial t^2}=\frac{\rho}{\epsilon_0}$$
This trick is called the Lorenz gauge.
    \section{Poisson and Laplace Equations}
\subsection{The Boundary Value Problem}
Many problems in mathematical physics can be reduced to the form:
$$\nabla^2\phi=F$$
where $F$ is known.
This form of equations are called Poisson's Equation.
In particular, if $F\equiv 0$, this is called Laplace's Equation.
We would want to solve them in either all of $\mathbb R^n$ or on some domain $\Omega\subset\mathbb R^n$.
We are, sometimes, only interested in the cases $n=2,3$.
Note that we require $\phi$ to be well-defined and smooth on all of $\Omega$.
For example, it is true that $\nabla^2(1/|\underline{x}|)=0$ for $\underline{x}\neq\underline{0}$, however it is not true for any domain containing $\underline{0}$.
We would want to solve this PDE subject to some boundary conditions, so $\phi$ will have predetermined behaviour in $\partial\Omega$ or as $|\underline{x}|\to\infty$ when working in $\mathbb R^n$.\\
Two widely studied boundary conditions are the Dirichlet Problem
$$\begin{cases}
    \nabla^2\phi=F\text{, on $\Omega$}\\
    f\text{, on $\partial\Omega$}
\end{cases}$$
and the Neumann Problem
$$\begin{cases}
    \nabla^2\phi=F\text{, on $\Omega$}\\
    \partial\phi/\partial\underline{n}=\underline{n}\cdot\nabla\phi=g\text{, on $\partial\Omega$}
\end{cases}$$
Beware that we must interpret boundary data (or boundary conditions) correctly.
We want $\phi$ or $\partial\phi/\partial\underline{n}$ to approach the boundary data continuously as $\underline{x}$ tends towards the boundary.
So apart from requiring $\phi$ to be $C^2$ in $\Omega$, it must also extend to $\partial\Omega$ continuously.
\begin{example}[Non-example]
    If we want to solve the Navier-Stokes Equation
    $$\frac{\partial\underline{u}}{\partial t}+(\underline{u}\cdot\nabla)\underline{u}-\nu\nabla^2\underline{u}=-\nabla p,\nabla\cdot\underline{u}=0,\underline{u}(\underline{x},0)=\underline{u_0}(\underline{x})$$
    but ignore the condition on continuous extension we said earlier, then the following solution satisfies the equation
    $$\underline{u}=\begin{cases}
        0\text{, if $t>0$}\\
        \underline{u_0}\text{, if $t=0$}
    \end{cases},p\equiv 0$$
    But obviously we are not getting a million for it.
\end{example}
\begin{example}
    Let $r=|\underline{x}|$, we consider the Dirichlet Problem
    $$\begin{cases}
        \nabla^2\phi=r\text{, for $r<a$}\\
        \phi=1\text{, for $r=a$}
    \end{cases}$$
    By symmetry, we want to write $\phi=\phi(r)$, so
    $$r=\nabla^2\phi=\frac{1}{r^2}\frac{\mathrm d}{\mathrm dr}\left( r^2\frac{\mathrm d\phi}{\mathrm dr} \right)\implies \phi=\frac{r^3}{12}-\frac{A}{r}+B$$
    The boundary condition then implies
    $$\phi(r)=1+\frac{r}{12}(r^2-a^2)$$
\end{example}
Consider now a generic linear problem, say $L\phi=F$ in $\Omega$ and $B\phi=f$ on $\partial\Omega$ where $L,B$ are linear differential operators.
Suppose $\phi_1,\phi_2$ are solutions to the system, then if we let $\psi=\phi_1-\phi_2$, we have $L\psi=B\psi=0$.
If we can show that this solves to $\psi=0$, then we know the uniqueness of the solution to our original equation.
So solution to a linear problem is unique iff the only solution to the corresponding homogeneous problem is $0$.
\begin{proposition}
    Solution to the Dirichlet Problem is unique.
    Solution to the Neumann Problem is unique up to a constant.
\end{proposition}
\begin{proof}
    Consider the homogeneous problem
    $$\begin{cases}
        \nabla^2\psi=0\text{, in $\Omega$}\\
        B\psi=0\text{, in $\partial\Omega$}
    \end{cases}$$
    where $B\psi=\psi$ in the Dirichlet case and $B\psi=\partial\psi/\partial\underline{n}$ in the Neumann case.
    Consider
    $$I[\phi]=\int_\Omega|\nabla\psi|^2\,\mathrm dV\ge 0$$
    But we have
    \begin{align*}
        I[\phi]&=\int_\Omega\nabla\cdot(\psi\nabla\psi)-\psi\nabla^2\psi\,\mathrm dV\\
        &=\int_\Omega\nabla\cdot(\psi\nabla\psi)\,\mathrm dV\\
        &=\int_{\partial\Omega}\psi\nabla\psi\cdot\mathrm d\underline{S}\\
        &=\int_{\partial\Omega}\psi\frac{\partial\phi}{\partial\underline{n}}\,\mathrm dS\\
        &=0
    \end{align*}
    in both cases, $\nabla\psi=0$ thoughout $\Omega$, so $\psi$ is continuous throughout $\Omega$.
    So for the Dirichlet Problem we have $\psi\equiv 0$ and $\psi$ is constant in the Neumann Problem.
\end{proof}
\begin{example}
    Consider the charge distribution (where $r=\underline{\underline{x}}$),
    $$\rho(\underline{x})=\begin{cases}
        0\text{, if $r<a$}\\
        F(r)\text{, if $r\ge a$}
    \end{cases}$$
    The corresponding potential $\phi$ for electric field $\underline{E}=-\nabla\phi$ would have
    $$\nabla^2\phi=-\epsilon_0^{-1}\rho$$
    On $r<a$, we have $\nabla^2\phi=0$, so by symmetry, we write $\phi=\phi(r)$.
    Note that on $r=a$, $\phi=\phi(a)$ is a constant.
    We can see that $\phi(r)=\phi(a)$ on $r<a$ actually works, but by the preceding proposition, it is the solution on $r<a$, so $\underline{E}\equiv \underline{0}$ on $r<a$.\\
    This looks like the Newton's Shell Theorem.
\end{example}
\subsection{Gauss's Flux Method}
There is a clever way to get particular solutions to Poisson's Equation when the forcing term has spherical symmetry.
Suppose the forcing term is in the form $F(r)$ where $r=|\underline{x}|$, and we are interested in a particular solution of the equation $\nabla^2\phi=F(r)$.
We want to look for solutions of the form $\phi=\phi(r)$, in which case $\nabla\phi=\phi^\prime(r)\underline{e_r}$.
If we integrate this over the ball $|\underline{x}|\le R$, then since $\nabla^2=\nabla\cdot\nabla$,
$$\int_{|\underline{x}|\le R}F\,\mathrm dV=\int_{|\underline{x}|\le R}\nabla^2\phi\,\mathrm dV=\int_{|\underline{x}|=R}\nabla\phi\cdot\mathrm d\underline{S}$$
by Divergence Theorem.
Note that $\mathrm d\underline{S}=\underline{e_r}\mathrm dS$, so
$$\int_{|\underline{x}|=R}\nabla\phi\cdot\mathrm d\underline{S}=\int_{|\underline{x}|=R}\phi^\prime(r)\,\mathrm dS=\phi^\prime(R)\int_{|\underline{x}|=R}\mathrm dS=4\pi R^2\phi^\prime(R)$$
Define
$$Q(R)=\int_{|\underline{x}|\le R}F(r)\,\mathrm dV$$
then $Q(R)=4\pi R^2\phi^\prime(R)$.
If $F$ is interpreted as the charge density, then we can interpret $Q(R)$ as total charge (or other stuff) inside the ball of radius $R$.
We can integrate this to get a particular solution.
In particular, we observe that $\phi^\prime(R)=Q(R)/(4\pi R^2)$, which is just the inverse square law.
\begin{example}
    Consider charge density
    $$\rho(r)=\begin{cases}
        \rho_0\text{, if $r\le a$}\\
        0\text{, otherwise}
    \end{cases}$$
    The electric field corresponding to $\rho$ satisfies $\nabla\cdot\underline{E}=\epsilon_0^{-1}\rho$.
    In electrostatics, $\underline{E}=-\nabla\phi$ for a potential $\phi$, so we have
    $$\nabla^2\phi=-\frac{\rho}{\epsilon_0}$$
    By previous calculations
    $$\phi^\prime(R)=-\frac{1}{4\pi\epsilon_0}\frac{Q(R)}{R^2},Q(R)=\begin{cases}
        4\pi R^3\rho_0/3\text{, if $R\le a$}\\
        Q=Q(a)=4\pi a^3\rho_0/3\text{, otherwise}
    \end{cases}$$
    So for $r>a$, we have
    $$\phi^\prime(r)=-\frac{1}{4\pi\epsilon_0}\frac{Q}{r^2}\implies\underline{E}(r)=\frac{1}{4\pi\epsilon_0}\frac{Q}{r^2}\underline{e_r}$$
    Let $a\to 0$ in such a way that $a^3\rho_0$ remains constant (so we change $\rho_0$), so $Q$ remains constant.
    Therefore the electric field induced by a point charge $Q$ at $\underline{x}=\underline{0}$ would have
    $$\underline{E}(\underline{x})=\frac{1}{4\pi\epsilon_0}\frac{Q}{r^2}\underline{e_r}=\frac{1}{4\pi\epsilon_0}\frac{Q}{|\underline{x}|^3}\underline{x},\phi(\underline{x})=\frac{1}{4\pi\epsilon_0}\frac{Q}{|\underline{x}|}$$
    for $r>0$ and $\phi(\infty)=0$.
\end{example}
Just an aside, if we want the existece of solutions to $-\nabla^2\phi=\epsilon_0^{-1}\rho(\underline{x})$, Gauss proved that it suffices to consider equations of the form $K\sigma=f$ which looks like linear algebra, but in a infinite dimensional vector spaces, where we no longer have sequential compactness.
This induces the study of functional analysis and operator theory.\\
Suppose we have $F=F(\rho)$ where $\rho=x^2+y^2$.
To solve $\nabla^2\varphi=F(\rho)$, it is natural to try solutions of the form $\varphi=\varphi(\rho)$.
In this case, $\nabla\varphi=\varphi^\prime(\rho)\underline{e_\rho}$, so obviously we want to integrate it over a cylinder $V$ of height $1$, $V=\{0\le\rho\le R,0\le\phi\le 2\pi,z_0\le z\le z_0+1\}$, as $\underline{e_\rho}\perp\underline{e_z}$,
$$Q(R)=\int_VF\,\mathrm dV=\int_{\partial V}\nabla\varphi\cdot\mathrm d\underline{S}=\varphi^\prime(R)2\pi R\implies\varphi^\prime(R)=\frac{Q(R)}{2\pi R}$$
Note that by calculation we have
$$Q(R)=2\pi\int_0^RF(\rho)\rho\,\mathrm d\rho\implies\varphi^\prime(\rho)=\frac{1}{R}\int_0^RF(\rho)\rho\,\mathrm d\rho$$
\begin{example}
    Due to conflicts of notation we write $s^2=x^2+y^2$ instead of $\rho$ (as we want to do electromagnetism).
    Suppose we have charge density
    $$\rho(s)=\begin{cases}
        \rho_0\text{, for $s\le a$}\\
        0\text{, otherwise}
    \end{cases}$$
    For electrostatic potential $\varphi$, we still have $-\nabla^2\varphi=\epsilon_0^{-1}\rho$, so
    $$\varphi^\prime(R)=-\frac{1}{\epsilon_0R}\int_0^R\rho(s)s\,\mathrm ds=\begin{cases}
        -\epsilon_0^{-1}R^{-1}\rho_0R^2/2\text{, if $R\le a$}\\
        -\epsilon_0^{-1}R^{-1}\rho_0a^2/2\text{, otherwise}\\
    \end{cases}$$
    Then shrink $a\to 0$ with $\rho_0a^2$ fixed, then we have $\underline{E}\propto s^{-1}\underline{e_s}$, which is similar to the superposition a bunch of point charges in a line.
\end{example}
\subsection{The Superposition Principle}
For a linear problem, we can (most of the time) solve them by their defining property of being linear.
Say if we have $L\psi_n=F_n$, then $L(\sum_n\psi_n)=\sum_nF_n$.
This allows us to superpose solutions.
If we write a forcing term $F$ as $\sum_nF_n$ and solve for $\psi_n$ individually, then we can obtain the required solution by summing up all of them.
\begin{example}
    1. Recall the solutions for an electric potential $\phi$ and electric field $\underline{E}=-\nabla\phi$.
    For charge distribution with spherical symmetry, can be found by e.g. Gauss's flux method.
    If we shrink the radius to $0$, we obtain a point charge $Q_{\underline{a}}$ for at $\underline{a}$ where
    $$\phi(\underline{x})=\frac{Q_{\underline{a}}}{4\pi\epsilon_0}\frac{1}{|\underline{x}-\underline{a}|}$$
    and $\rho(\underline{x})=Q_{\underline{a}}\delta(\underline{x}-\underline{a})$.
    For consistency with Gauss's Law, we shall (and indeed can) obtain
    $$\nabla^2\left( -\frac{1}{4\pi}\frac{1}{|\underline{x}-\underline{a}|} \right)=\delta(\underline{x}-\underline{a})$$
    Now consider the electric potential due to $2$ charges $Q_{\underline{a}},Q_{\underline{b}}$ at $\underline{x}=\underline{a},\underline{b}$.
    The charge distribution would be $Q_{\underline{a}}\delta(\underline{x}-\underline{a})+Q_{\underline{b}}\delta(\underline{x}-\underline{b})$, so we can superpose the solutions correspondingly in Gauss's Law to get the potential
    $$\phi(\underline{x})=\frac{Q_{\underline{a}}}{4\pi\epsilon_0}\frac{1}{|\underline{x}-\underline{a}|}+\frac{Q_{\underline{b}}}{4\pi\epsilon_0}\frac{1}{|\underline{x}-\underline{b}|}$$
    2. We want to find the potential outside the solid sphere $|\underline{x}|\le R$ of uniform charge density $\rho_0$, from which several spheres $|\underline{x}-\underline{a_i}|\le R_i$ with $i=1,2,\ldots,n$ are removed (given that the spheres do not cross the boundary).
    To find the solution, we can superpose the solution for the charge distribution $\rho_0$ for $|\underline{x}|\le R$ and the solutions for the charge distribution $-\rho_0$ for $|\underline{x}-\underline{a_i}|\le R_i$.
    So
    $$\phi(\underline{x})=\frac{1}{4\pi\epsilon_0}\left( \frac{Q}{|\underline{x}|}-\sum_{i=1}^n\frac{Q_i}{|\underline{x}-\underline{a_i}|} \right),Q=\frac{4}{3}\pi R^3\rho_0,Q_i=\frac{4}{3}\pi R_i^3\rho_0$$
    for $|\underline{x}|>R$.
\end{example}
\subsection{Integral Solutions}
In the examples above, we found solutions by superposing (or superimposing) potentials corresponding to charges in different points, which gives solutions of the form
$$\sum_i\frac{Q_i}{4\pi\epsilon_0}\frac{1}{|\underline{x}-\underline{a_i}|}$$
This leads to a more general form of superposition of potentials, by thinking of each infinitesimal part as individual point charges, which is just a integral in the following form:
$$\int_{\mathbb R^3}\frac{F(\underline{y})}{|\underline{x}-\underline{y}|}\,\mathrm dV(\underline{y})$$
Up to some factor.
\begin{proposition}
    The unique solution to the Dirichlet problem
    $$\begin{cases}
        \nabla^2\phi=F\text{, in $\mathbb R^3$}\\
        \phi(\underline{x})\to 0\text{ as $|\underline{x}|\to\infty$}
    \end{cases}$$
    (Assuming $F$ decreases sufficiently rapidly as $|\underline{x}|\to\infty$) is
    $$\phi(\underline{x})=-\frac{1}{4\pi}\int_{\mathbb R^3}\frac{F(\underline{y})}{|\underline{x}-\underline{y}|}\,\mathrm dV$$
\end{proposition}
\begin{proof}
    The solution can be verified by using
    $$\nabla\left( -\frac{1}{4\pi}\frac{1}{|\underline{x}-\underline{a}|} \right)=\delta(\underline{x}-\underline{a})$$
    And differentiating under the integral sign.
\end{proof}
To justify (in an applied way, of course) the identity used above, we use the divergence theorem,
\begin{align*}
    \int_{|\underline{x}|\le R}\nabla^2\left( \frac{1}{|\underline{x}|} \right)\,\mathrm dV&=\int_{|\underline{x}|=R}\nabla\left( \frac{1}{r} \right)\cdot\mathrm d\underline{S}\\
    &=-\frac{1}{R^2}\int_{|\underline{x}|=R}\underline{e_r}\cdot\underline{e_r}\,\mathrm dS\\
    &=\frac{1}{R^2}4\pi R^2\\
    &=4\pi
\end{align*}
which is true for any $R>0$, so it is natural (or maybe not) to write the identity.\\
The solution can be regarded as the sum of contributions to the potential to any possible volume elements (that actually contribute).
\subsection{Harmonic Functions}
If $\phi=\phi(\underline{x})$ satisfies Laplace's Equation $\nabla^2\phi=0$, then $\phi$ is harmonic.
\begin{proposition}[Mean-Value Property]
    If $\varphi$ is harmonic on some open $\Omega\subset\mathbb R^3$, then
    $$\varphi(\underline{a})=\frac{1}{4\pi r^2}\int_{|\underline{x}-\underline{a}|=r}\varphi(\underline{x})\,\mathrm dS$$
    where $\underline{a}\in\Omega$ and the ball centered at $\underline{a}$ with radius $r$ is contained in $\Omega$.
\end{proposition}
\begin{proof}
    Define a function
    \begin{align*}
        f(r)&=\frac{1}{4\pi r^2}\int_{|\underline{x}-\underline{a}|=r}\varphi(\underline{x})\,\mathrm dS\\
        &=\frac{1}{4\pi r^2}\int_{|\underline{x}|=r}\varphi(\underline{x}+\underline{a})\,\mathrm dS\\
        &=\frac{1}{4\pi r^2}\int_0^{2\pi}\int_0^\pi \varphi(\underline{a}+r\underline{e_r})r^2\sin\theta\,\mathrm d\theta\,\mathrm d\phi\\
        &=\frac{1}{4\pi}\int_0^{2\pi}\int_0^\pi \varphi(\underline{a}+r\underline{e_r})\sin\theta\,\mathrm d\theta\,\mathrm d\phi
    \end{align*}
    So differentiating this gives
    \begin{align*}
        f^\prime(r)&=\frac{1}{4\pi}\int_0^{2\pi}\int_0^\pi \nabla\varphi(\underline{a}+r\underline{e_r})\cdot\underline{e_r}\sin\theta\,\mathrm d\theta\,\mathrm d\phi\\
        &=\frac{1}{4\pi r^2}\int_{|\underline{x}|=r}\nabla\varphi(\underline{a}+\underline{x})\cdot\mathrm d\underline{S}\\
        &=\frac{1}{4\pi r^2}\int_{|\underline{x}|\le r}\nabla^2\varphi(\underline{a}+\underline{x})\,\mathrm dV\\
        &=0
    \end{align*}
    So $f$ is constant.
    Let $r\to 0$ gives $f\equiv \varphi(\underline{a})$.
\end{proof}
\begin{proposition}
    For a smooth function $\varphi$
    $$\nabla^2\varphi=\lim_{r\to 0}\frac{6}{r^2}\left(\frac{1}{4\pi r^2}\int_{|\underline{x}-\underline{a}|=r}\varphi(\underline{x})\,\mathrm dS-\varphi(\underline{a})\right)$$
\end{proposition}
\begin{proof}
    Consider the function
    $$g(r)=\frac{1}{4\pi r^2}\int_{|\underline{x}-\underline{a}|=r}\varphi(\underline{x})\,\mathrm dS-\varphi(\underline{a})$$
    then use the same trick as above, we have
    $$g^\prime(r)=\frac{1}{4\pi r^2}\int_{|\underline{x}|\le r}\nabla^2\varphi(\underline{x})\,\mathrm dV$$
    But
    \begin{align*}
        \int_{|\underline{x}|\le r}\nabla^2\varphi(\underline{x})\,\mathrm dV&=\int_{|\underline{x}|\le r}\nabla^2\varphi(\underline{a})\,\mathrm dV+\left( \int_{|\underline{x}|\le r}\nabla^2\varphi(\underline{x})-\nabla^2\varphi(\underline{a})\,\mathrm dV \right)\\
        &=\frac{4\pi r^3}{3}\nabla^2\varphi(\underline{a})+o(r^3)
    \end{align*}
    as $r\to 0$.
    So
    $$g^\prime(r)=\frac{r}{3}\nabla^2\varphi(\underline{a})+o(r)$$
    But then $g^\prime(r)=g^\prime(0)+rg^\prime(0)+o(r)$ by Taylor's Theorem, so $g^\prime(0)=0,g^{\prime\prime}(0)=3^{-1}\nabla^2\varphi(\underline{a})$.
    So
    $$g(r)=g(0)+rg^\prime(0)+r^2\frac{g^{\prime\prime}(0)}{2}+o(r^2)=\frac{r^2}{6}\nabla^2\varphi(\underline{a})+o(r^2)$$
    So taking the stated limit gives the solution.
\end{proof}
So the Laplacian measures how much the value of the function at the point differs from the average of the values on the infinitesimal sphere centered at the same point.
\begin{proposition}[Maximum Principle]
    If $\varphi$ is harmonic on an open, path-connected $\Omega\in\mathbb R^3$, then there is an $\underline{a}\in\Omega$ such that $\varphi(\underline{x})\le\varphi(\underline{a})$ throughout $\Omega$, then $\varphi$ is constant.
\end{proposition}
\begin{proof}
    If $\varphi$ is harmonic, then the mean-value property holds, so
    $$\varphi(\underline{a})=\frac{1}{4\pi\epsilon^2}\int_{|\underline{x}-\underline{a}|=\epsilon}\varphi(\underline{x})\,\mathrm dS$$
    for $\epsilon$ sufficiently small.
    If $\varphi(\underline{x})\le\varphi(\underline{a})$ throughout $\Omega$, then
    $$0=\frac{1}{4\pi\epsilon^2}\int_{|\underline{x}-\underline{a}|=\epsilon}\varphi(\underline{a})-\varphi(\underline{x})\,\mathrm dS$$
    But the integrand is nonnegative, so we must have $\varphi(\underline{x})=\varphi(\underline{a})$ for any $\underline{x}$ in the sphere we are integrating, hence the ball enclosed.
    Now take any $\underline{y}\in\Omega$ and consider a path joining $\underline{a}$ and $\underline{y}$.
    Then by compactness, there is a finite collection of spheres in $\Omega$ whose interior covers the path.
    But then each adjascent two of them will share at least one interior points, and if $f$ achieve $f(\underline{a})$ at that point, then $f(\underline{x})\le f(\underline{a})$ for any $\underline{x}$ in each ball, so use the argument inductively gives $f(\underline{y})=f(\underline{a})$.
\end{proof}
One can also prove the last part without introducing compactness:
Assuming that we have already shown that for any $\underline{b}\in\Omega$ with $\varphi(\underline{b})=\alpha=\varphi(\underline{a})$, there is some $\epsilon>0$ such that $\varphi(\underline{x})=\varphi(\underline{b})=\alpha$ for any $|\underline{x}-\underline{b}|<\epsilon$ (by e.g. the first part of our proof above).
Consider the set $\varphi^{-1}(\{\alpha\})$, which is closed by continuity and open by above, which contradicts the path-connectedness (hence connectedness) of $\Omega$.
    \subsection{Bonus: Discrete Laplacian}
If we have $\varphi:\mathbb Z^n\to\mathbb Z$, we can define
$$\nabla^2\varphi(\underline{a})=\frac{1}{2n}\sum_{i=1}^n(\varphi(\underline{a}+\underline{e_i})+\varphi(\underline{a}-\underline{e_i}))-\phi(\underline{a})$$
We can show that if $\varphi$ is harmonic and bounded, then it is constant.
Indeed the set $\phi(\mathbb Z^n)$ is finite if $\phi$ is bounded, hence has a least element.
It is then trivial to show, by following the grid, that $\phi$ attains the value of the least element everywhere.
    \section{Cartesian Tensors}
In this section, we are only interested in the Cartesian coordinates in a right-handed basis.
\subsection{A Closer Look at Vectors}
Given the basis, we can write $\underline{x}\in\mathbb R^3$ in the form $x_i\underline{e_i}$ where the summation convention is used.
We should not identify the vector $\underline{x}$ with component $x_i$ since we may want to choose another basis for certain purposes.
But if we have $\underline{x}=x_i\underline{e_i}=x_i'\underline{e_i'}$ for right-handed bases $\underline{e_i},\underline{e_i'}$, then
$$x_i'=x_j'\delta_{ij}=x_j'\underline{e_j'}\cdot\underline{e_i'}=\underline{e_i'}\cdot(x_j'\underline{e_j'})=\underline{e_i'}\cdot(x_j\underline{e_j})=x_j\underline{e_i'}\cdot\underline{e_j}=R_{ij}x_j,R_{ij}=\underline{e_i'}\cdot\underline{e_j}$$
Playing the same game yields $x_i=\underline{e_i}\cdot\underline{e_j'}x_j'=R_{ji}x_j'$.
Combining them gives $x_i=R_{ji}R_{jk}x_k$, so $0=(R_{ji}R_{jk}-\delta_{ik})x_k$, which has to hold for any choice of $x_k$, therefore $R_{ji}R_{jk}=\delta_{ik}$.
If we set $R$ to be the matrix with entries $R_{ij}$, then what we obtained above means $R^\top R=I$.
So $R\in\operatorname{O}(3)$.
Now $x_i\underline{e_i}=x_i'\underline{e_i'}=R_{ij}x_j\underline{e_i'}=R_{ji}x_i\underline{e_j'}$, so $\underline{e_i}=R_{ji}\underline{e_j'}$, so $R$ has to be in $\operatorname{SO}(3)$ since both bases are right-handed.
In summary, changing from $\{\underline{e_i}\}$ to $\{\underline{e_i'}\}$ induces the change in components by multiplication of a matrix in $\operatorname{SO}(3)$, i.e. $x_i'=R_{ij}x_j$.
We call such objects rank-$1$ tensors of vectors.
\subsection{A Closer Look at Scalars}
Consider $\sigma=\underline{a}\cdot\underline{b}$.
Using $\underline{e_i}$ with $\underline{a}=a_i\underline{e_i},\underline{b}=b_i\underline{e_i}$, then $\sigma=a_ib_j\underline{e_i}\cdot\underline{e_j}=a_ib_j\delta{ij}=a_ib_i$.
If we used another set of basis vectors $\underline{a}=a_i'\underline{e_i'},\underline{b}=b_i'\underline{e_i'}$, then $\sigma'=a_i'b_i'$ has
$$\sigma'=a_i'b_i'=R_{ij}a_jR_{ik}b_k=R_{ij}R_{ik}a_jb_k=\delta_{jk}a_jb_k=a_jb_j=\sigma$$
as one may have expected.
Call such transformation of scalars rank-$0$ tensors.
\subsection{A Closer Look at Linear Maps}
Consider linear map $T:\mathbb R^3\to\mathbb R^3$ with $\underline{x}\mapsto\underline{y}=T(\underline{x})=\underline{x}-(\underline{x}\cdot\underline{n})\underline{n}$ which is jus the projection of $\underline{x}$ down the plane with normal $\underline{n}$.
If we use basis $\{\underline{e_i}\}$, then
$$y_i\underline{e_i}=T(x_j\underline{e_j})=x_jT(\underline{e_j})=x_j(\underline{e_j}-n_in_j\underline{e_i})=x_j(\delta_{ij}-n_in_j)\underline{e_i}$$
So $y_i=T_{ij}x_j$ where $T_{ij}=\delta_{ij}-n_in_j$.
If we used another basis $\underline{e_i'}$, we would have got $y_i'=T_{ij}'x_j'$ where $T_{ij}'=\delta_{ij}-n_i'n_j'$.
Note that using $n_i'=R_{ip}n_p$, etc., we have
$$T_{ij}'=\delta{ij}-R_{ip}R_{jq}n_pn_q=R_{ip}R_{jq}(\delta_{pq}-n_pn_q)=R_{ip}T_{pq}R_{jq}$$
So changing from one set of right handed orthonormal basis to another induces change in components of linear map $T$ (as a matrix) by $T_{ij}'=R_{ip}R_{jq}T_{pq}$, or $T'=RTR^\top$.
We call objects changing like that to be rank-$2$ tensors.
\subsection{Cartesian Tensors of General Rank}
\begin{definition}
    An object with components $T_{i_1\cdots i_n}$ is called a tensor of rank $n$ if its component transform according to $T_{i_1\cdots i_n}'=R_{i_1j_1}\cdots R_{i_nj_n}T_{j_1\cdots j_n}$ when we change from one right-handed Cartesian basis $\{\underline{e_i}\}$ to $\{\underline{e_i'}\}$ where $\det R=1$ and $R_{i_pi_r}R_{i_qi_r}=\delta_{i_pi_q}$ for $p,q,r$ distinct.
\end{definition}
Note that $R_{ij}$'s are rotation matrices.
\begin{example}
    1. If $u_{i_1},v_{i_2},\ldots,w_{i_n}$ are components of set of $n$ vectors, then $T_{i_1\cdots i_k}=u_{i_1}v_{i_2}\cdots w_{i_n}$ is a tensor of rank $n$.
    Suppose we change from $\{\underline{e_i}\}$ to $\{\underline{e_i'}\}$, then
    $$T_{i_1\cdots i_n}'=u_{i_1}'v_{i_2}'\cdots w_{i_n}'=R_{i_1j_1}u_{j_1}R_{i_2j_2}u_{j_2}\cdots R_{i_nj_n}u_{j_n}=R_{i_1j_1}\cdots R_{i_nj_n}T_{j_1\cdots j_n}$$
    2. The Kronecker delta $\delta_{ij}$ is a tensor of rank $2$ as it is independent of choice of basis.
    Indeed, we want $R_{ip}R_{jq}\delta_{pq}=R_{ip}R_{jp}=\delta_{ij}=\delta_{ij}'$.\\
    3. The Levi-Civita epsilon again is independent of choice of basis, so $\epsilon_{ijk}'=\epsilon_{ijk}$.
    We have
    $$R_{ip}R_{jq}R_{kr}\epsilon_{pqr}=\det(R)\epsilon_{ijk}=\epsilon_{ijk}=\epsilon_{ijk}'$$
    So it is a tensor of rank $3$.\\
    4. Experimental evidence suggests a linear relationship between code $j$ produced in some medium that is exposed to electric field $\underline{E}$.
    So a given Cartesian basis $\{\underline{e_i}\}$, we must have numbers $\sigma_{ij}$ such that $J_i=\sigma_{ij}E_j$, so if we change basis from $\{\underline{e_i}\}$ to $\{\underline{e_i'}\}$, then $\sigma_{ij}'E_j'=J_i'=R_{ip}J_p=R_{ip}\sigma_{pq}E_q=R_{ip}R_{jq}\sigma_{pq}E_j'$, so $\sigma_{ij}'=R_{ip}R_{jq}\sigma_{pq}$.
    So $\sigma$ is a tensor of rank $2$.
    This is an example of something called the quotient theorem, which will be proved at the end of the course.
\end{example}
\begin{example}[Non-example]
    Not every array of numbers is a tensor.
    For example, in some given basis $\{\underline{e_i}\}$ we define an array
    $$(a_{ij})=\begin{pmatrix}
        1&2&3\\
        4&5&6\\
        7&8&\pi
    \end{pmatrix}$$
    and $a_{ij}=0$ in any other choice of basis, then $a_{ij}$ is not the component of a second rank tensor.
\end{example}
\begin{definition}
    If $a,b$ are rank-$n$ tensors with components $a_{i_1\cdots i_n},b_{i_1\cdots i_n}$, then the object $a+b$ by $(a+b)_{i_1\cdots i_n}=a_{i_1\cdots i_n}+b_{i_1\cdots i_n}$ is also a tensor of rank $n$.
    If $\alpha$ is a scalar, then we can define the tensor $\alpha a$ by $(\alpha a)_{i_1\cdots i_n}=\alpha a_{i_1\cdots i_n}$.
\end{definition}
\begin{definition}
    If $U$ is a tensor of rank $n$ and $V$ a tensor of rank $m$, then their tensor product $U\otimes V$ is a tensor of rank $m+n$ defined by
    $$(U\otimes V)_{i_1\cdots i_nj_1\cdots j_m}=U_{i_1\cdots i_n}V_{j_1\cdots j_m}$$
\end{definition}
\begin{definition}
    Suppose $n\ge 2$ and $T$ is a tensor of rank $n$, we can define a new tensor of rank $n-2$ by contraction on two indices (i.e. summing over two chosen indices).
\end{definition}
It is easy to check that these are indeed all tensors.
We say $T_{i_1\cdots i_n}$ is symmetric in $i_1,i_2$ if $T_{i_1i_2\cdots i_n}=T_{i_2i_1\cdots i_n}$.
This is obviously well-behaved.
Note that we can generalize it to symmetries in any pair of indices.
We say it is antisymmetric in $i_1,i_2$ if $T_{i_1i_2\cdots i_n}=-T_{i_2i_1\cdots i_n}$.
We say it is totally symmetric if it is symmetric in each pair of indices, and totally antisymmetric if it is antisymmetric in any two indices.
\begin{example}
    Both $\delta_{ij}$ and $a_ia_ja_k$ are totally symmetric tensors.
    Also $\epsilon_{ijk}$ is totally antisymmetric.
    In fact, one can see immediately that the Levi-Civita $\epsilon$ is the only antisymmetric tensor of rank $3$ up to proportionality.
\end{example}
Also, there are no nonzero totally antisymmetric tensor of rank $n\ge 4$ in $\mathbb R^3$.
\subsection{Tensor Calculus}
We say $T_{i_1\cdots i_n}(\underline{x})$ is a tensor field of rank $n$ if for each $\underline{x_0}\in\mathbb R^3$, $T_{i_1\cdots i_n}(\underline{x_0})$ is a tensor of rank $n$.
Note that $x_i'=R_{ij}x_j$ when we transform to one orthonormal right-handed basis to the other.
Also $x_j=R_{kj}x_k'$.
By the chain rule, $\partial/\partial x_i'=(\partial x_j/\partial x_i')(\partial/\partial x_j)=R_{ij}\partial/\partial x_j$.
\begin{proposition}
    Suppose $T_{i_1\cdots i_n}(\underline{x})$ is a tensor field of rank $n$, then
    $$A_{j_1\cdots j_mi_1\cdots i_n}(\underline{x})=\frac{\partial}{\partial x_{j_1}}\cdots\frac{\partial}{\partial x_{j_m}}T_{i_1\cdots i_n}(\underline{x})$$
    is a tensor field of rank $m+n$.
\end{proposition}
\begin{proof}
    From definition and chain rule.
\end{proof}
\begin{example}
    1. If $\phi$ is a scalar field, then components of $\nabla\phi$ changes according to $[\nabla\phi]_i'=\partial\phi/\partial x_i'=R_{ij}\partial\phi/\partial x_j=R_{ij}[\nabla\phi]_j$, so $\nabla\phi$ is a rank $1$ tensor field (or vector field).\\
    2. If $\underline{v}$ is a vector field, then using the same trick we can see that $\nabla\cdot\underline{v}$ is a rank $0$ tensor (or scalar field).\\
    3. If $\underline{v}$ is a vector field, so is $\nabla\times\underline{v}$.
\end{example}
\begin{example}
    Recall the divergence theorem for vector fields:
    $$\int_V\nabla\cdot\underline{F}\,\mathrm dV=\int_{\partial V}\underline{F}\cdot\mathrm dS$$
    Equivalently (or not),
    $$\int_V\frac{\partial F_i}{\partial x_i}\,\mathrm dV=\int_{\partial V}v_in_i\,\mathrm dS$$
    Turns out we can do this on tensor fields as well, where we have
    $$\int_V\frac{\partial}{\partial x_{i_k}}T_{i_1\cdots i_n}\,\mathrm dV=\int_{\partial V}T_{i_1\cdots i_n}n_{i_k}\,\mathrm dS$$
    which follows from the case for vector fields on the field
    $$v_{i_k}=a_{i_1}b_{i_2}\cdots c_{i_n}T_{i_1\cdots i_k\cdots i_n}$$
\end{example}
\subsection{Tensors of Rank 2}
An arbitrary rank-$2$ tensor $T_{ij}$ can be written as
$$T_{ij}=\frac{1}{2}(T_{ij}+T_{ji})+\frac{1}{2}(T_{ij}-T_{ji})=S_{ij}+A_{ij}$$
So $S_{ij}$ is symmetric and $A_{ij}$ is antisymmetric.
Note that $S_{ij}$ only has $6$ independent components, and $A_{ij}$ has $3$ independent components.
This is all consistent since a rank $3$ tensor has $9=6+3$ independent components.
\begin{proposition}
    A rank $2$ tensor can be written as $T_{ij}=S_{ij}+\epsilon_{ijk}\omega_k$ where $S_{ij}$ is symmetric and $\omega_k=\epsilon_{kpq}T_{pq}/2$.
    Also, this decomposition is unique.
\end{proposition}
\begin{proof}
    Just expand by taking $S_{ij}=(T_{ij}+T_{ji})/2$ for existence.
    As for uniqueness, suppose $S_{ij}+\epsilon_{ijk}\omega_k=\tilde{S}_{ij}+\epsilon_{ijk}\tilde\omega_k$.
    But we can take the symmetric part of each sides to get $S_{ij}=\tilde{S}_{ij}$, hence $\omega_k=\tilde\omega_k$.
\end{proof}
\begin{example}
    Suppose each point $\underline{x}$ in an elastic body undergoes small displacement $\underline{u}(\underline{x})$, then consider two points, initially seperated by $\delta\underline{x}$, then after the displacement they are seperated by
    $$\underline{u}(\underline{x}+\delta\underline{x})+\underline{x}+\delta\underline{x}-\underline{u}(\underline{x})-\underline{x}=\delta\underline{x}+(\underline{u}(\underline{x}+\delta\underline{x})-\underline{u}(\underline{x}))$$
    So the change in seperation would be $\underline{\underline{x}+\delta\underline{x}}-\underline{u}(\underline{x})$.
    Now using Cartesian and suffix notation, we have $u_i(\underline{x}+\delta\underline{x})-u_i(\underline{x})=\delta x_j\partial u_i/\partial x_j+o(|\delta\underline{x}|)$
    We write $\partial u_i/\partial x_j=e_{ij}+\epsilon_{ijk}\omega_k$ where
    $$e_{ij}=\frac{1}{2}\left(\frac{\partial u_i}{\partial x_j}+\frac{\partial u_j}{\partial x_i}\right),\omega_k=\frac{1}{2}\epsilon_{kpq}\partial u_p/\partial x_q=-\frac{1}{2}[\nabla\times\underline{u}]_k$$
    $e_{ij}$ here is called the linear strain tensor.
    So we have
    $$\underline{u}(\underline{x}+\delta\underline{x})-\underline{u}(\underline{x})=e_{ij}\delta x_j+[\delta\underline{x}\times\underline{\omega}]_i+o(|\delta\underline{x}|)$$
    So $e_{ij}$ tells you how the material strains,
\end{example}
Suppose a body occupies a volume $V$ has density $\rho(\underline{x})$ and suppose each point is rotating with angular velocity $\underline{\omega}$ through the origin, then the velocity of the point $\underline{x}$ is $\underline{\omega}\times\underline{x}$.
Then the total angular momentum is
$$\underline{L}=\int_V\rho(\underline{x})(\underline{x}\times\underline{v})\,\mathrm dV=\int_V\rho(\underline{x})(\underline{x}\times(\underline{\omega}\times\underline{x}))\,\mathrm dV$$
Using a right-handed basis $\{\underline{e_i}\}$ of $\mathbb R^3$, we have
$$L_i=\int_{\mathcal V}\rho(\underline{x})(x_kx_k\omega_i-x_ix_j\omega_j)\,\mathrm dV=I_{ij}\omega_j,I_{ij}=\int_{\mathcal V}\rho(\underline{x})(x_kx_k\delta_{ij}-x_ix_j)\,\mathrm dV$$
where $\mathcal V=\{(x_1,x_2,x_3):\underline{x}=x_i\underline{e_i}\in V\}$.
If we change our basis to another $\{\underline{e_i'}\}$, then we have (with $x_i'=R_{ij}x_j$)
\begin{align*}
    I_{ij}'&=\int_{\mathcal V'}\rho(\underline{x})(x_k'x_k'\delta_{ij}-x_i'x_j')\,\mathrm dV\\
    &=R_{ip}R_{jq}\int_{\mathcal V}\rho(\underline{x})(x_kx_k\delta_{pq}-x_px_q)|J|\,\mathrm dV\\
    &=R_{ip}R_{jq}I_{pq}
\end{align*}
So $I_{ij}$ is really a tensor.
We call it the inertial tensor.
\begin{example}
    Consider an ellipsoid
    $$\frac{x_1^2}{a^2}+\frac{x_2^2}{b^2}+\frac{x_3^2}{c^2}=1$$
    with $\rho(\underline{x})\equiv\rho_0$.
    By symmetry, if $i\neq j$, then $I_{ij}=0$.
    Now
    $$I_{11}=\rho_0\int_Vx_2^2x_3^2\,\mathrm dV$$
    By using scaled spherical polars $x_1=ar\cos\phi\sin\theta,x_2=br\sin\phi\sin\theta,x_3=cr\cos\theta$, so $\mathrm dV=abcr^2\sin\theta\,\mathrm dr\,\mathrm d\theta\,\mathrm d\phi$, we have
    $$I_{11}=\int_0^{2\pi}\int_0^\pi\int_0^1 r^2(b^2\sin^2\phi\sin^2\theta+c^2\cos^2\theta)abcr^2\sin\theta\,\mathrm dr\,\mathrm d\theta\,\mathrm d\phi=\frac{M}{5}(b^2+c^2)$$
    So
    $$(I_{ij})=\frac{M}{5}\begin{pmatrix}
        b^2+c^2&0&0\\
        0&a^2+c^2&0\\
        0&0&a^2+b^2
    \end{pmatrix}$$
    If in particular $a=b=c$, then $I_{ij}\propto\delta_{ij}$.
\end{example}
\begin{proposition}
    If $T_{ij}$ is real and symmetric, then there exists choice of basis in which $T_{ij}=0$ whenever $i\neq j$.
\end{proposition}
\begin{proof}
    In Vectors \& Matrices.
\end{proof}
\subsection{Isotropic Tensors}
\begin{definition}
    Say $T_{i_1\cdots i_n}$ is isotropic if $T_{i_1\cdots i_n}'=T_{i_1\cdots i_n}$ when transforming from a basis to the other.
    That is, for any rotational $R$, we have
    $$T_{i_1\cdots i_n}=R_{i_1j_1}\cdots R_{i_nj_n}T_{j_1\cdots j_n}$$
\end{definition}
\begin{example}
    1. Scalars are isotropic.\\
    2. $\delta_{ij}$ is isotropic.\\
    3. $\epsilon_{ijk}$ is also isotropic.
\end{example}
It turns out that we can classify all isotropic tensors in $\mathbb R^3$, and we can generalise this to $\mathbb R^n$.
We state this in the proposition below, which we shall provide a partial proof.
\begin{proposition}
    In $\mathbb R^3$:\\
    1. All scalars are isotropic.\\
    2. There are no nonzero isotropic rank-$1$ tensors (vectors).\\
    3. Most general isotropic tensor of rank $2$ is $\alpha\delta_{ij}$ where $\alpha$ is a scalar.\\
    4. Most general isotropic tensor of rank $3$ is $\beta\epsilon_{ijk}$ where $\beta$ is a scalar.\\
    5. Most general isotropic tensor of rank $4$ is $\alpha\delta_{ij}\delta_{kl}+\beta\delta_{il}\delta_{jk}+\gamma\delta_{ik}\delta_{jl}$ where $\alpha,\beta,\gamma$ are scalars.\\
    6. Tensors of higher rank is a linear combination of $\epsilon$'s and $\delta$'s, e.g. $\delta_{ij}\epsilon_{pqr}$ is an isotropic rank $5$ tensor.
\end{proposition}
\begin{proof}
    1 is obvious.\\
    For 2, assume $v_i$ is an isotropic tensor of rank $1$, then $v_i=R_{ij}v_j$ for all choice of rotation $R$.
    If we choose
    $$(R_{ij})=\begin{pmatrix}
        -1&0&0\\
        0&-1&0\\
        0&0&1
    \end{pmatrix}$$
    then we immediately get $v_1=v_2=0$.
    Similarly $v_3=0$, so $v=0$.\\
    For 3, suppose $T_{ij}$ is isotropic, then $T_{ij}=R_{ip}R_{jq}T_{pq}$ for any rotation $R$.
    Choose
    $$(R_{ij})=\begin{pmatrix}
        0&1&0\\
        -1&0&0\\
        0&0&1
    \end{pmatrix}$$
    Then
    $$T_{23}=R_{2p}R_{3q}T_{pq}=R_{21}R_{33}T_{13}=-T_{13},T_{13}=R_{1p}R_{3q}T_{pq}=R_{12}R_{33}T_{23}=T_{23}$$
    so we conclude $T_{13}=T_{23}=0$.
    Now $T_{11}=R_{1p}R_{1q}T_{pq}=T_{22}$, so $T_{11}=T_{22}$.
    Consider another rotation matrix
    $$(R_{ij})=\begin{pmatrix}
        1&0&0\\
        0&0&1\\
        0&-1&0
    \end{pmatrix}$$
    then
    $$T_{31}=R_{3p}R_{1q}T_{pq}=R_{32}R_{11}R_{21}=-R_{21},T_{21}=R_{2p}R_{1q}T_{pq}=R_{23}R_{11}T_{31}=T_{31}$$
    So $T_{31}=T_{21}=0$.
    Lastly
    $$T_{32}=R_{3p}R_{2q}T_{pq}=R_{32}R_{23}T_{23}=0,T_{12}=R_{1p}R_{2q}T_{pq}=R_{11}R_{23}T_{13}=0$$
    So in conclusion $T_{ij}=0$ whenever $i\neq j$
    Also $T_{33}=R_{3p}R_{3q}T_{pq}=R_{32}R_{32}T_{22}=T_{22}$, so $T_{11}=T_{22}=T_{33}$.
    Hence $T_{ij}=T_{11}\delta_{ij}$.
    Take $\alpha=T_{11}$ completes the proof.\\
    4,5,6 can be proved by similar idea.
\end{proof}
Consider tensors of the form
$$T_{i_1\cdots i_n}=\int_{V_R}f(r)x_{i_1}\cdots x_{i_n}\,\mathrm dV$$
where $r^2=x_px_p$ and $V_R$ is a ball of radius $R$ centered at $0$.
Then when we go to another frame of reference by $R$,
$$T_{i_1\cdots i_n}'=R_{i_1j_1}\cdots R_{i_nj_n}T_{j_1\cdots j_n}=R_{i_1j_1}\cdots R_{i_nj_n}\int_{V_R}f(r')x_{j_1}\cdots x_{j_n}\,\mathrm dV$$
where $r'^2=r^2$ since $R$ is a rotation.
Set $y_{i_k}=R_{i_kj_k}x_{j_k}$ and do a change of variable in this way, we get
$$T_{i_1\cdots i_n}'=\int_{V_R}f(r')y_{i_1}\cdots y_{i_n}\,\mathrm dV=T_{i_1\cdots i_n}$$
Since $V_R$ is indeendent of rotation.
So $T$ is indeed isotropic.
Taking $R\to\infty$ allows us to view it as an integration over $\mathbb R^3$.
\begin{example}
    Consider
    $$T_{ij}=\int_{\mathbb R^3}e^{-r^5}x_ix_j\,\mathrm dV=T_{11}\delta_{ij}$$
    by our classification theorem.
    Also $T_{ii}=4\pi/5$, so $T_{ij}=4\pi\delta_{ij}/15$.
\end{example}
\begin{example}
    The inertial tensor of a ball with radius $R>0$ and constant density $\rho_0$, so
    $$I_{ij}=\rho_0\int_{V_R}x_kx_k\delta_{ij}-x_kx_j\,\mathrm dV$$
    The right hand side is the sum of two isotropic tensor of rank $2$, so $I_{ij}=\alpha\delta_{ij}$.
    Contract on $i,j$ gives $\alpha=2MR^2/5$ where $M=\rho_0 4\pi R^3/3$.
\end{example}
\subsection{Multilinear Maps and Quotient Theorem}
Given some right-handed orthonormal basis, let $T_{ij}$ denote the components of rank $2$ tensor.
Define a bilinear map $t:\mathbb R^3\times\mathbb R^3\to\mathbb R$ by $t(\underline{a},\underline{b})=T_{ij}a_ib_j$, which one can note is independent of the basis we chose hence the map is well-defined.
Conversely, for a bilinear $t:\mathbb R^3\times\mathbb R^3\to\mathbb R$, and we choose a certain basis $\{\underline{e_i}\}$, then we can write $t(\underline{a},\underline{b})=a_ib_jt(\underline{e_i},\underline{e_j})$, so $T_{ij}=t(\underline{e_i},\underline{e_j})$ is a rank-$2$ tensor since  $t$ is bilinear.
This gives a one-to-one correspondence between bilinear maps and rank-$2$ tensors.
In particular, if a (bilinear) map $(\underline{a},\underline{b})\mapsto T_{ij}a_ib_j$ is well-defined, then $T_{ij}$ is naturally a tensor.\\
In general, we can correspondingly identify a rank-$n$ tensor in $\mathbb R^3$ by multilinear maps $(\mathbb R^3)^n\to\mathbb R$.
\begin{proposition}[Quotient Theorem]
    Given basis $\{\underline{e_i}\}$, let $T_{i_1\cdots i_nj_1\cdots j_m}$ be array of numbers such that $v_{i_1\cdots i_n}=T_{i_1\cdots i_nj_1\cdots j_m}u_{j_1\cdots j_m}$ is a tensor for any tensor $u$, then $T$ is a tensor.
\end{proposition}
\begin{proof}
    Take $u_{j_1\cdots j_m}=c^1_{j_1}\cdots c^m_{j_m}$, where $\underline{c^k}$ are vectors.
    So
    $$v_{i_1\cdots i_n}=T_{i_1\cdots i_nj_1\cdots j_m}c^1_{j_1}\cdots c^m_{j_m}$$
    is a tensor by hypothesis.
    Let $\underline{a^1},\ldots,\underline{a^n}$ be vectors, then we can contract $v$ by $v_{i_1\cdots i_n}a_{i_1}^1\cdots a_{i_n}^n$, which is a scalar that is independent of basis, so the map
    $$(\underline{a^1},\ldots,\underline{a^n},\underline{c^1},\ldots,\underline{c^n})\mapsto T_{i_1\cdots i_nj_1\cdots j_m}a_{i_1}^1\cdots a_{i_n}^nc_{j_1}^1\cdots c_{j_m}^m$$
    is independent of choice of coordinates, so $T$ is a tensor.
\end{proof}
\begin{example}
    Recall linear strain tensor $e_{ij}=(\partial u_i/\partial x_j+\partial u_j/\partial x_i)/2$ where $\underline{u}(\underline{x})$ is the displacement of the particle at $\underline{x}$ of a body undergoing deformation.
    Experimental evidence suggests a linear relationship between stresses (internal forces) and strain.
    We measure stress using stress tensor $\sigma_{ij}$.
    There are $3^4=81$ numbers $c_{ijkl}$ such that $\sigma_{ij}=c_{ijkl}e_{kl}$.
    This is just a generalization of Hookes' Law to higher dimensions.
    Now if we know that $c_{ijkl}=c_{ijlk}$ we know that $c_{ijkl}$ is a tensor from the quotient theorem.
    In this case we call this array $c_{ijkl}$ is the stiffness tensor.
    For isotropic material, we know that $c_{ijkl}=\alpha\delta_{ij}\delta_{kl}+\beta\delta_{ik}\delta_{jl}+\gamma\delta_{il}\delta_{jk}$, so $\sigma_{ij}=\alpha e_{kk}\delta_{ij}+\beta e_{ij}+\gamma e_{ji}=\alpha e_{kk}\delta_{ij}+2\mu e_{ij}$ where $\mu=\beta+\gamma$.
    We can invert for $e_{ij}$ by contract on indices $i,j$, so $\sigma_{kk}=(3\alpha+2\mu)e_{kk}$, so $e_{kk}=\sigma_{kk}/(3\alpha+2\mu)$.
    So $2\mu e_{ij}=\sigma_{ij}-\alpha\sigma_{kk}\delta_{ij}/(3\alpha+2\mu)$.
\end{example}
\end{document}