\section{Coordinates, Differentials and Gradients}
\subsection{Differentials and First-Order Changes}
Recall that if $f=f(u_1,\ldots,u_n)$, then we write $\mathrm df=(\partial f/\partial u_i)\,\mathrm du_i$ (the summation convention is being used).
Those $\mathrm du_i$ are formal objects called differential forms which are quite abstract geometrical notions that are way beyond the scope of this course.
These differential forms are taken as linearly independent the same way as vectors are.
Similarly, if $\underline{x}=\underline{x}(u_1,u_2,\ldots,u_n)$, then $\mathrm d\underline{x}=(\partial \underline{x}/\partial u_i)\,\mathrm du_i$.
\begin{example}
    If $f(u,v,w)=u^2-v^2+e^w$, then $\mathrm df=2u\,\mathrm du-2v\,\mathrm dv+e^w\,\mathrm dw$.
    If $\underline{x}=(u^2,v^2,w^2)^\top$, then $\mathrm d\underline{x}=(2u\,\mathrm du,2v\,\mathrm dv,2w\,\mathrm dw)^\top$.
\end{example}
Differential forms give a great tool to describe first-order changes.
If we perturb a multivariable function $f(u_1,\ldots,u_n)$, then we can have
$$f(u_1+\epsilon_1,\ldots,u_n+\epsilon_n)=f(u_1,\ldots,u_n)+\frac{\partial f}{\partial u_i}\epsilon_i+o(\|\underline{\epsilon}\|)$$
We can get the chain rule ``for free'' by using this notion.
Suppose we change our coordinates by $v_i=v_i(u_1,\ldots,u_n)$ and $F(u_1,\ldots,u_n)=f(v_1,\ldots,v_n)$, so
$$\frac{\partial F}{\partial u_i}\,\mathrm du_i=dF=df=\frac{\partial f}{\partial v_j}\,\mathrm dv_j=\frac{\partial f}{\partial v_j}\frac{\partial v_j}{\partial u_i}\,\mathrm du_i$$
Therefore
\begin{theorem}
    $$\frac{\partial F}{\partial u_i}=\frac{\partial f}{\partial v_j}\frac{\partial v_j}{\partial u_i}$$
\end{theorem}
Note that the summation convention is implicitly used.
\subsection{Coordinates in Line Elements}
Say $u,v$ are coordinates for $\mathbb R^2$ by relating them to Cartesians in the form $x=x(u,v),y=y(u,v)$ such that these smooth functions can be inverted smoothly to give $u=u(x,y),v=v(x,y)$.
\begin{example}
    Consider the polar coordinate $(r,\theta)$ with the relationship $x=r\cos\theta,y=r\sin\theta$.
    We can invert to have $r=\sqrt{x^2+y^2},\tan\theta=x/y$.
    \footnote{This is not quite invertible at $(x,y)=(0,0)$.
    Just saying.}
\end{example}
\begin{example}
    1. The Cartesian coordinate in $\mathbb R^2$ is $\underline{x}=\underline{x}(x,y)=(x,y)^\top$.
    \footnote{For future reference, any column vector in this course are implicitly written as per the standard basis unless otherwise specified.}
    Note that $\underline{x}_x,\underline{x}_y$ give the standard basis, so $\mathrm d\underline{x}=(\mathrm dx,\mathrm dy)^\top$.\\
    2. The polar coordinate defined above, $\underline{x}=(r\cos\theta,r\sin\theta)^\top$ has
    $$\underline{x}_r=(\cos\theta,\sin\theta),\underline{x}_\theta=(-r\sin\theta,r\cos\theta)$$
    which becomes an orthonormal basis (which depends on $(r,\theta)$) if we normalize.
    So the line element has
    $$\mathrm d\underline{x}=\begin{pmatrix}
        \cos\theta\\\sin\theta
    \end{pmatrix}\,\mathrm dr+r\begin{pmatrix}
        -\sin\theta\\\cos\theta
    \end{pmatrix}\,\mathrm d\theta$$
    So small change in the line element can result in a large change in the coordinate $\mathrm d\theta$.
    The basis vectors above give the rotation basis.
\end{example}
\begin{definition}
    We say $u,v,w$ are set of orthogonal curvilinear coordinates for $\mathbb R^3$ if the unit vectors $\underline{e}_u=\underline{x}_u/\|\underline{x}_u\|,\underline{e}_v=\underline{x}_v/\|\underline{x}_v\|,\underline{e}_w=\underline{x}_w/\|\underline{x}_w\|$ always forms a right-handed system of orthonormal vectors.
\end{definition}
\begin{definition}
    The scale factors are
    $$h_{u}=\left|\frac{\partial \underline{x}}{\partial u}\right|,h_{v}=\left|\frac{\partial \underline{x}}{\partial v}\right|,h_{w}=\left|\frac{\partial \underline{x}}{\partial w}\right|$$
\end{definition}
So the factors are the scaling factors for a little change in the corresponding coordinates.
\begin{definition}
    The cylindral polar coordinates are $(\rho,\phi,z)$ where
    $$\underline{x}=\underline{x}(\rho,\phi,z)=(\rho\cos\phi,\rho\sin\phi,z)^\top$$
\end{definition}
\begin{definition}
    The spherical polar coordinates are $(r,\theta,\phi)$ where
    $$\underline{x}=\underline{x}(r,\theta,\phi)=(r\cos\phi\sin\theta, r\sin\phi\sin\theta,r\cos\theta)$$
\end{definition}
\subsection{The Gradient Operator}
For $f:\mathbb R^3\to\mathbb R$, define the gradientof $f$, $\nabla f$, by
$$f(\underline{x}+\underline{h})=f(\underline{x})+\nabla f(\underline{x})\cdot\underline{h}+o(\underline{h})$$
\begin{definition}
    The directional derivative of $f$ in direction $\underline{v}$ by
    $$D_{\underline{v}}f(\underline{x})=\lim_{t\to 0}\frac{f(\underline{x}+t\underline{v})-f(\underline{x})}{t}$$
    So $f(\underline{x}+t\underline{v})=f(\underline{x})+tD_{\underline{v}}f(\underline{x})+o(t)$.
\end{definition}
So if we let $\underline{h}=t\underline{v}$, we have
$$f(\underline{x}+t\underline{v})=f(\underline{x})+\nabla f(\underline{x})\cdot t\underline{v}+o(t)$$
So we have $\nabla f(\underline{x})\cdot \underline{v}=D_{\underline{v}}f(\underline{x})$.
By Cauchy-Schwartz, to maximize a dot product we will have to make the two vectors parallel, so
\begin{proposition}
    $\nabla f(\underline{x})$ is the direction of greatest increase of $f$ at $\underline{x}$.
\end{proposition}
\begin{proof}
    Cauchy-Schwartz.
\end{proof}
\begin{example}
    1. $f(\underline{x})=|\underline{x}|^2/2$, so
    $$f(\underline{x}+\underline{h})=(\underline{x}+\underline{h})\cdot(\underline{x}+\underline{h})/2=f(\underline{x})+\underline{x}\cdot\underline{h}+o(\underline{h})$$
    Hence $\nabla f(\underline{x})=\underline{x}$.\\
    2. For generic curve $t\mapsto \underline{x}(t)$ and a function $F$, we want to evaluate $(F\circ \underline{x})^\prime$, so
    $$F(\underline{x}(t+\delta t))=F(\underline{x}(t)+\underline{\delta x})=F(\underline{x}(t))+\nabla f(\underline{x}(t))\cdot\underline{\delta x}+o(\underline{\delta x})$$
    where $\underline{\delta x}=\underline{x}(t+\delta t)-\underline{x}(t)=t\underline{x}^\prime(t)+o(t)$.
    So plugging it in we have
    $$\frac{\mathrm dF}{\mathrm dt}=\nabla F(\underline{x}(t))\cdot \underline{x}^\prime(t)$$
    3. Consider a surface in $\mathbb R^3$ by $S=\{\underline{x}:f(\underline{x})=0\}$ where $f:\mathbb R^3\to\mathbb R$.
    Take curve $t\mapsto \underline{x}(t)$ such that $\underline{x}\in S$ for all $t$, then $0=\underline{x}^\prime(t)\cdot\nabla F(\underline{x}(t))$, so $\nabla F$ is perpendicular to the tangent to the curve.
    Hence necessarily $\nabla F$ is the normal to $S$.
\end{example}
\subsection{Computing the Gradient}
For general orthogonal curvilinear coordinate, it might be hard to calculate $\nabla f$ since we do not usually know how to change the coordinates to accomodate the change $\underline{h}$.\\
But it is easy in Cartesians.
Just evaluating the directional derivatives at the basis vectors reveals:
\begin{proposition}
    $$\nabla f=\begin{pmatrix}
        \partial f/\partial x\\
        \partial f/\partial y\\
        \partial f/\partial z
    \end{pmatrix}$$
\end{proposition}
\begin{proof}
    We have
    $$(\nabla f(\underline{x}))_i=\nabla f(\underline{x})\cdot\underline{e_i}=D_{\underline{e_i}}f(\underline{x})$$
    As desired.
\end{proof}
\begin{example}
    Again we take $f(\underline{x})=|\underline{x}|^2/2$, then $(\nabla f(\underline{x}))_i=x_i$, so $\nabla f(\underline{x})=\underline{x}$ as before.
\end{example}
In Cartesians, we know the line elements $\mathrm d\underline{x}=\mathrm dx_i\underline{e_i}$ which allows us to calculate easily.
But we have $\mathrm df=D_{\underline{e_i}}f\,\mathrm dx_i$ in any coordinate.
So immediately we have
\begin{proposition}
    $\mathrm df=\nabla f\cdot\mathrm d\underline{x}$
\end{proposition}
which is coordinate independent.
\begin{proof}
    Just calculate the right hand side.
\end{proof}
\begin{proposition}
    Let $u,v,w$ be a set of curvilinear coordinates, then we have
    $$\nabla f=\frac{1}{h_u}\frac{\partial f}{\partial u}\underline{e_u}+\frac{1}{h_v}\frac{\partial f}{\partial v}\underline{e_v}+\frac{1}{h_w}\frac{\partial f}{\partial w}\underline{e_w}$$
\end{proposition}
\begin{proof}
    $\mathrm df=\nabla f\cdot\mathrm d\underline{x}$ independent of coordinate.
    We also know that $\mathrm d\underline{x}=h_u\underline{e_u}\,\mathrm du+h_v\underline{e_v}\,\mathrm dv+h_w\underline{e_w}\,\mathrm dw$.
    Write $\nabla f=(\nabla f)_u\underline{e_u}+(\nabla f)_v\underline{e_v}+(\nabla f)_w\underline{e_w}$, then
    $$\nabla f\cdot \mathrm d\underline{x}=h_u(\nabla f)_u\,\mathrm du+h_v(\nabla f)_v\,\mathrm dv+h_w(\nabla f)_w\,\mathrm dw$$
    In addition,
    $$\mathrm df=\frac{\partial f}{\partial u}\,\mathrm du+\frac{\partial f}{\partial v}\,\mathrm dv+\frac{\partial f}{\partial w}\,\mathrm dw$$
    But they are equal.
    Since $\mathrm du,\mathrm dv,\mathrm dw$ are linearly independent, we have
    $$(\nabla f)_u=\frac{1}{h_u}\frac{\partial f}{\partial u},(\nabla f)_v=\frac{1}{h_v}\frac{\partial f}{\partial v},(\nabla f)_w=\frac{1}{h_w}\frac{\partial f}{\partial w}$$
    As desired.
\end{proof}
\begin{example}
    1. For cylindral coordinates $(\rho,\phi,z)$, we have
    $$\nabla f=\frac{\partial f}{\partial\rho}\underline{e_\rho}+\frac{1}{\rho}\frac{\partial f}{\partial\phi}\underline{e_\phi}+\frac{\partial f}{\partial z}\underline{e_z}$$
    Then in the previous example where $f(\underline{x})=|\underline{x}|^2/2=(\rho^2+z^2)/2\implies\nabla f=\rho\underline{e_\rho}+z\underline{e_z}=\underline{x}$\\
    2. For spherical coordinates $(r,\theta,\phi)$, we can do the same thing,
    $$\nabla f=\frac{\partial f}{\partial r}\underline{e_r}+\frac{1}{r}\frac{\partial f}{\partial\theta}\underline{e_\theta}+\frac{1}{r\sin\theta}\frac{\partial f}{\partial \phi}\underline{e_\phi}$$
    Using the same example $f(\underline{x})=r^2/2\implies \nabla f=r\underline{e_r}=\underline{x}$.
\end{example}
\begin{note}
    We talked about the functions about position vectors under different coordinates,
    $$f(\underline{x})=f(\underline{x}(x,y,z))=f(\underline{x}(r,\theta,\phi))$$
    So when we are talking about $f$, sometimes we are telling
    $$\tilde{f}(x,y,z)=f(\underline{x}(x,y,z)),\tilde{\tilde{f}}(r,\theta,\rho)=f(\underline{x}(r,\theta,\rho))$$
    Or other coordinate we might find interesting.
    We are actually talking about a pullback here which might be clear in a couple of years' time.
\end{note}